\documentclass[handout]{beamer}
\usepackage{beamerthemesplit}
\usepackage{pgfpages}
\usepackage{verbatim}
\usepackage{amsmath, amssymb, graphics, setspace}
\usepackage{wasysym}

\usepackage{algorithm}
\usepackage{algorithmicx}
\usepackage{algpseudocode}

\usepackage{tikz}
\usepackage{circuitikz}
\usetikzlibrary{dsp,chains}

\DeclareMathAlphabet{\mathpzc}{OT1}{pzc}{m}{it}
\newcommand{\z}{\mathpzc{z}}


\newcommand{\field}[1]{\mathbb{#1}} %requires amsfonts

\usetheme{Antibes}
\usecolortheme{beaver}
\title[Thesis Progress Report \#7]{Thesis Progress Report \#7}

\usepackage{mathptmx}
\usepackage[scaled=.90]{helvet}
\usepackage{courier}
\usepackage[T1]{fontenc}

%\pgfpagesuselayout{4 on 1}[letterpaper,border shrink=5mm]

\institute[RIT]{}
\date{\today}
%\subtitle{}
\author{Christopher A. Wood}
%\institute[]{}
\date{May 29, 2013}

\begin{document}

%%%%%
%%
%% Resource link: http://www.math-linux.com/spip.php?article77
%%
%%%%

\begin{frame}
	\titlepage
\end{frame}

\begin{frame}
	\frametitle{Agenda}
	\tableofcontents
\end{frame}

\section{Updates}
\begin{frame}
	\frametitle{Updates}
	\begin{itemize}
		\item Magma?
	\end{itemize}
\end{frame}

\section{Inverse in Polynomial and Normal Basis}
\begin{frame}
	\frametitle{Counting the number of gates}
	Main idea:
	\begin{itemize}
		\item Decompose operations in $GF(2^{16})$ to $GF(2)$.
		\item Implement $GF(2)$ operations using simple logic gates.
	\end{itemize}

	\medskip

	Key
	\begin{itemize}
		\item $GF(2^{16})/GF(2^8)$ : $s(y) = y^2 + \Psi y + \Lambda$
		\item $GF(2^8)/GF(2^4)$ : $r(x) = x^2 + \Theta x + \Pi$
		\item $GF(2^4) / GF(2^2)$ : $q(w) = w^2 + \Omega w + \Sigma$
		\item $GF(2^2) / GF(2)$ : $p(v) = v^2 + \Gamma v + \Delta$ ($\Gamma = 1$, $\Delta = 1$)
		\item $\theta \in GF(2^{16})$, $\zeta \in GF(2^8)$, $\epsilon \in GF(2^4)$, $\delta \in GF(2^2)$, $\gamma \in GF(2)$
	\end{itemize}
%% INVERSE IN GF(2^16) using composite fields
\begin{figure}
%\centering
\end{figure}
\end{frame}

\begin{frame}[fragile]
	\frametitle{Multiplicative inverse in $GF(2^{16})$}
	\begin{align*}
		\zeta^{-1} = (\epsilon_1 y + \epsilon_2)^{-1}
	\end{align*}
\begin{figure}
\begin{tikzpicture}

% Place nodes using a matrix
	\matrix (m1) [row sep=2.5mm, column sep=5mm]
	{
		%--------------------------------------------------------------------
		\node[coordinate]                  (m00) {};          &
		\node[coordinate]                  (m01) {};          &
		\node[coordinate]                  (m02) {};          &
		\node[coordinate]                  (m03) {};          &
		\node[coordinate]                  (m04) {};          &
		\node[coordinate]                  (m05) {};          &
		\node[coordinate]                  (m06) {};          \\

		\node[dspsquare]                   (m10) {$\epsilon_1$}; &
		\node[coordinate]                  (m11) {};          &
		\node[dspsquare]                   (m12) {$\Lambda \otimes \epsilon_1^2$};          &
		\node[coordinate]                  (m13) {};          &
		\node[coordinate]                  (m14) {};          &
		\node[dspmixer]                   (m15) {};          &
		\node[dspsquare]                  (m16) {$\epsilon_3$};         \\

		\node[coordinate]                   (m20) {}; &
		\node[dspadder]                  (m21) {};          &
		\node[dspnodefull]                  (m22) {};          &
		\node[dspadder]                   (m23) {};          &
		\node[dspsquare]                  (m24) {$\epsilon^{-1}$};          &
		\node[coordinate]                   (m25) {};          &
		\node[coordinate]                  (m26) {};         \\

		\node[dspsquare]                   (m30) {$\epsilon_2$}; &
		\node[coordinate]                  (m31) {};          &
		\node[dspmixer]                    (m32) {};          &
		\node[coordinate]                  (m33) {};          &
		\node[coordinate]                  (m34) {$\epsilon^{-1}$};          &
		\node[dspmixer]                    (m35) {};          &
		\node[dspsquare]                   (m36) {$\epsilon_4$};         \\
	};

	\begin{scope}[start chain]
		\chainin (m10);
		\chainin (m12) [join=by dspconn];
		\chainin (m23) [join=by dspconn];
		\chainin (m24) [join=by dspconn];
		\chainin (m15) [join=by dspconn];
		\chainin (m16) [join=by dspconn];
	\end{scope}

	\begin{scope}[start chain]
		\chainin (m10);
		\chainin (m01) [join=by dspline];
		\chainin (m02) [join=by dspline];
		\chainin (m03) [join=by dspline];
		\chainin (m04) [join=by dspline];
		\chainin (m15) [join=by dspconn];
	\end{scope}

	\begin{scope}[start chain]
		\chainin (m10);
		\chainin (m21) [join=by dspconn];
		\chainin (m22) [join=by dspline];
		\chainin (m33) [join=by dspline];
		\chainin (m35) [join=by dspconn];
		\chainin (m36) [join=by dspconn];
	\end{scope}

	\begin{scope}[start chain]
		\chainin (m30);
		\chainin (m21) [join=by dspconn];
	\end{scope}

	\begin{scope}[start chain]
		\chainin (m30);
		\chainin (m32) [join=by dspconn];
		\chainin (m24) [join=by dspconn];
	\end{scope}

	\begin{scope}[start chain]
		\chainin (m24);
		\chainin (m35) [join=by dspconn];
	\end{scope}

	\begin{scope}[start chain]
		\chainin (m22);
		\chainin (m32) [join=by dspconn];
	\end{scope}
\end{tikzpicture}
\end{figure}

\end{frame}

\begin{frame}[fragile]
	\frametitle{Multiplication in $GF(2^8)$}
	\begin{align*}
		\zeta_1 \times \zeta_2 = (\epsilon_1 y + \epsilon_2) \times (\epsilon_3 y + \epsilon_4) = (\epsilon_5 y + \epsilon_6)
	\end{align*}
\begin{figure}
	\begin{tikzpicture}

% Place nodes using a matrix
	\matrix (m1) [row sep=2.5mm, column sep=5mm]
	{
		%--------------------------------------------------------------------
		\node[dspsquare]                  (m00) {$\epsilon_1$};          &
		\node[coordinate]                  (m01) {};          &
		\node[coordinate]                  (m02) {};          &
		\node[dspmixer]                  (m03) {};          &
		\node[dspsquare]                  (m04) {$\times \Pi$};          &
		\node[coordinate]                  (m05) {};          &
		\node[coordinate]                  (m06) {};          \\

		\node[dspsquare]                  (m10) {$\epsilon_2$};          &
		\node[dspadder]                  (m11) {};          &
		\node[coordinate]                  (m12) {};          &
		\node[coordinate]                  (m13) {};          &
		\node[coordinate]                  (m14) {};          &
		\node[dspadder]                  (m15) {};          &
		\node[dspsquare]                  (m16) {$\epsilon_5$};          \\

		\node[coordinate]                  (m20) {};          &
		\node[coordinate]                  (m21) {};          &
		\node[coordinate]                  (m22) {};          &
		\node[dspmixer]                  (m23) {};          &
		\node[coordinate]                  (m24) {};          &
		\node[coordinate]                  (m25) {};          &
		\node[coordinate]                  (m26) {};          \\

		\node[dspsquare]                  (m30) {$\epsilon_3$};          &
		\node[dspadder]                  (m31) {};          &
		\node[coordinate]                  (m32) {};          &
		\node[coordinate]                  (m33) {};          &
		\node[coordinate]                  (m34) {};          &
		\node[dspadder]                  (m35) {};          &
		\node[dspsquare]                  (m36) {$\epsilon_6$};          \\

		\node[dspsquare]                  (m40) {$\epsilon_4$};          &
		\node[coordinate]                  (m41) {};          &
		\node[coordinate]                  (m42) {};          &
		\node[dspmixer]                  (m43) {};          &
		\node[coordinate]                  (m44) {};          &
		\node[coordinate]                  (m45) {};          &
		\node[coordinate]                  (m46) {};          \\
	};

	\begin{scope}[start chain]
		\chainin (m00);
		\chainin (m03) [join=by dspconn];
		\chainin (m04) [join=by dspconn];
		\chainin (m35) [join=by dspconn];
		\chainin (m36) [join=by dspconn];
	\end{scope}

	\begin{scope}[start chain]
		\chainin (m00);
		\chainin (m11) [join=by dspconn];
		\chainin (m23) [join=by dspconn];
		\chainin (m14) [join=by dspline];
		\chainin (m15) [join=by dspconn];
		\chainin (m16) [join=by dspconn];
	\end{scope}

	\begin{scope}[start chain]
		\chainin (m10);
		\chainin (m11) [join=by dspconn];
	\end{scope}

	\begin{scope}[start chain]
		\chainin (m10);
		\chainin (m43) [join=by dspconn];
	\end{scope}

	\begin{scope}[start chain]
		\chainin (m30);
		\chainin (m31) [join=by dspconn];
		\chainin (m23) [join=by dspconn];
	\end{scope}

	\begin{scope}[start chain]
		\chainin (m30);
		\chainin (m03) [join=by dspconn];
	\end{scope}

	\begin{scope}[start chain]
		\chainin (m40);
		\chainin (m31) [join=by dspconn];
	\end{scope}

	\begin{scope}[start chain]
		\chainin (m40);
		\chainin (m43) [join=by dspconn];
		\chainin (m44) [join=by dspline];
		\chainin (m35) [join=by dspline];
		\chainin (m16) [join=by dspconn];
	\end{scope}

	\begin{scope}[start chain]
		\chainin (m44);
		\chainin (m35) [join=by dspconn];
	\end{scope}

\end{tikzpicture}
\end{figure}

\end{frame}

\begin{frame}[fragile]
	\frametitle{Squaring in $GF(2^8)$}
	\begin{align*}
		\zeta^2 = (\epsilon_1 y + \epsilon_2)(\epsilon_1 y + \epsilon_2) = (\epsilon_3 y + \epsilon_4)
	\end{align*}
\begin{figure}
\centering
\begin{tikzpicture}

% Place nodes using a matrix
	\matrix (m1) [row sep=2.5mm, column sep=5mm]
	{
		%--------------------------------------------------------------------
		\node[dspsquare]                  (m00) {$\epsilon_1$};          &
		\node[dspsquare]                  (m01) {$\epsilon_1^2$};          &
		\node[coordinate]                  (m02) {};          &
		\node[coordinate]                  (m03) {};          &
		\node[dspsquare]                  (m04) {$\epsilon_3$};          \\

		\node[coordinate]                   (m10) {}; &
		\node[coordinate]                  (m11) {};          &
		\node[dspsquare]                    (m12) {$\times \Pi$};          &
		\node[coordinate]                  (m13) {};          &
		\node[coordinate]                  (m14) {};          \\

		\node[dspsquare]                   (m20) {$\epsilon_2$}; &
		\node[dspsquare]                  (m21) {$\epsilon_2^2$};          &
		\node[coordinate]                    (m22) {};          &
		\node[dspadder]                    (m23) {};          &
		\node[dspsquare]                  (m24) {$\epsilon_4$};          \\
	};

	\begin{scope}[start chain]
		\chainin (m00);
		\chainin (m01) [join=by dspconn];
		\chainin (m02) [join=by dspline];
		\chainin (m03) [join=by dspline];
		\chainin (m04) [join=by dspconn];
	\end{scope}

	\begin{scope}[start chain]
		\chainin (m01);
		\chainin (m12) [join=by dspconn];
		\chainin (m23) [join=by dspconn];
	\end{scope}

	\begin{scope}[start chain]
		\chainin (m20);
		\chainin (m21) [join=by dspconn];
		\chainin (m22) [join=by dspline];
		\chainin (m23) [join=by dspconn];
		\chainin (m24) [join=by dspconn];
	\end{scope}
\end{tikzpicture}
\end{figure}
\end{frame}

\begin{frame}[fragile]
	\frametitle{Scaling in $GF(2^8)$}
	\begin{align*}
		\zeta \times \Lambda = (\epsilon_1 y + \epsilon_2)\times \Lambda = (\epsilon_3 y + \epsilon_4)
	\end{align*}
	\begin{figure}
\centering
\begin{tikzpicture}

% Place nodes using a matrix
	\matrix (m1) [row sep=2.5mm, column sep=5mm]
	{
		%--------------------------------------------------------------------
		\node[dspsquare]                  (m00) {$\epsilon_1$};          &
		\node[dspadder]                  (m01) {};          &
		\node[dspsquare]                  (m02) {$\times \Pi^{-1}$};          &
		\node[dspsquare]                  (m03) {$\epsilon_3$};          \\

		\node[dspsquare]                   (m10) {$\epsilon_2$}; &
		\node[coordinate]                  (m11) {};          &
		\node[coordinate]                    (m12) {};          &
		\node[dspsquare]                  (m13) {$\epsilon_4$};          \\
	};

	\begin{scope}[start chain]
		\chainin (m00);
		\chainin (m01) [join=by dspconn];
		\chainin (m02) [join=by dspconn];
		\chainin (m03) [join=by dspconn];
	\end{scope}

	\begin{scope}[start chain]
		\chainin (m00);
		\chainin (m11) [join=by dspline];
		\chainin (m13) [join=by dspconn];
	\end{scope}

	\begin{scope}[start chain]
		\chainin (m10);
		\chainin (m01) [join=by dspconn];
	\end{scope}
\end{tikzpicture}
\end{figure}
\end{frame}

\begin{frame}
	\frametitle{Subfield operations}
	\begin{center}
	Perform the algebra, minimize the arithmetic, create the circuit, count the gates  
	\end{center}
\end{frame}

\begin{frame}
	\frametitle{Basis changes to degree 2 extension tower field representations}
	\begin{center}
	Magma!
	\end{center}
\end{frame}

%Optimizations

\section{Linear Transformation Optimizations}
\begin{frame}
	\frametitle{Optimizing linear transformations}
\begin{align*}
\begin{pmatrix}
y_0 \\
y_1 \\
y_2 \\
y_3 \\
y_4 \\
y_5 \\
y_6 \\
y_7 \\
\end{pmatrix} = 
\begin{pmatrix}
1 & 1 & 0 & 1 & 1 & 1 & 0 & 0 \\
0 & 1 & 1 & 1 & 1 & 1 & 1 & 0 \\
0 & 1 & 0 & 1 & 1 & 1 & 0 & 0 \\
1 & 1 & 0 & 1 & 0 & 0 & 1 & 0 \\
1 & 0 & 0 & 1 & 1 & 0 & 1 & 0 \\
1 & 1 & 0 & 1 & 1 & 0 & 1 & 0 \\
1 & 1 & 0 & 1 & 0 & 0 & 0 & 0 \\
1 & 1 & 1 & 1 & 1 & 1 & 1 & 1 \\
\end{pmatrix}
\begin{pmatrix}
x_0 \\
x_1 \\ 
x_2 \\
x_3 \\
x_4 \\
x_5 \\
x_6 \\
x_7 \\
\end{pmatrix}
\end{align*}
\end{frame}

\begin{frame}
	\frametitle{Equivalently...}
\begin{align*}
\begin{pmatrix}
y_0 \\
y_1 \\
y_2 \\
y_3 \\
y_4 \\
y_5 \\
y_6 \\
y_7 \\
\end{pmatrix} = 
\begin{pmatrix}
x_0 + x_1 + x_3 + x_4 + x_5 \\
x_1 + x_2 + x_3 + x_4 + x_5 + x_6 \\
x_1 + x_3 + x_4 + x_5 \\
x_0 + x_1 + x_3 + x_6 \\
x_0 + x_3 + x_4 + x_6 \\
x_0 + x_1 + x_3 + x_4 + x_6 \\
x_0 + x_1 + x_3 \\
x_0 + x_1 + x_3 + x_4 + x_5 + x_6 + x_7 \\
\end{pmatrix}
\end{align*}
\end{frame}

\begin{frame}
	\frametitle{A better solution}
\begin{align*}
\begin{pmatrix}
y_0 \\
y_1 \\
y_2 \\
y_3 \\
y_4 \\
y_5 \\
y_6 \\
y_7 \\
\end{pmatrix} = 
\begin{pmatrix}
x_9 + x_{10} \\
x_6 + x_8 + x_{12} \\
x_8 + x_{10} \\
x_{11} \\
x_0 + x_3 + x_4 + x_6 \\
x_4 + x_{11} \\
x_9 \\
x_7 + x_{11} + x_{12} \\
\end{pmatrix}
\end{align*}

with $x_8 = x_1 + x_3$, $x_9 = x_0 + x_8$, $x_{10} = x_4 + x_5$, $x_{11} = x_6 + x_9$, and $x_{12} = x_2 + x_{10}$

\end{frame}

\begin{frame}
	\frametitle{Optimization}
	Main idea:
	\begin{itemize}
		\item Factor out bits that can share a gate
		\item Cast the new shared gate as a new variable in the linear system
		\item Repeat factorization until no more shares can be found
	\end{itemize}

	\medskip 

	Fun fact:
	\begin{itemize}
		\item This is formalized as the \emph{Shortest Linear Program} problem - NP-hard
		\item Complexity proved by considering the decision variant - does there exist a linear program with at most $k$ lines which computes the function? - and reducing from VERTEX-COVER
		\item Approximation algorithms have ratios of at least $3/2$ (i.e. they cannot yield near-optimal solutions)
	\end{itemize}
\end{frame}

\begin{frame}
	\frametitle{Optimization}
	Morale: heuristics are needed!
	\begin{itemize}
		\item Paar: Greedy factoring
		\item Peralta: Greedy factoring with (Euclidean) norm-based tie breaker (details discussed in the thesis)
	\end{itemize}

	\medskip 

	Question: which one works best for $16 \times 16$ matrices (linear transformations)?
\end{frame}

\begin{frame}
	\frametitle{Exhaustive search for the optimal linear program}
	... or, perform an exhaustive search similar to Canright

	\medskip

	\textbf{Our contribution}: reimplement Canright's exhaustive search and then \emph{parallelize} it

	\medskip

	\textbf{Problem}: Exhaustive search is recursive, so how can we partition the work among multiple cores/threads? \frownie

	\medskip

	\textbf{Solution}: \emph{fork/join recursive actions} \smiley
\end{frame}

\begin{frame}
	\frametitle{Demonstration}
	\begin{center}
		Demo (in Java)
	\end{center}
\end{frame}

\section{Other Maximally Nonlinear Power Mappings}
\begin{frame}
	\frametitle{Cryptographically significant power mappings}
	All S-boxes are based off of some \emph{nonlinear power mapping}: $f(x) = x^d$

	\begin{tabular}{c || c }
    \hline
    \textbf{Name} & \textbf{Exponent} (d)  \\ \hline
    Inverse & $-1 \equiv 2^n - 2$  \\ \hline
    Gold & $2^{k} + 1$, $\gcd\{k,n\} = 1$ for some $1 \leq k \leq 2^{n} - 1$  \\ \hline
    Kasami & $2^{2k} - 2^k + 1$, $\gcd\{k,n\} = 1$ for some $1 \leq k \leq 2^{n-1} - 1$ \\ \hline
    Dobertin & $2^{4k + 3k + 2k + k} - 1$ over $GF(2^s)$ with $s = 5k$  \\ \hline
    Niho & $2^{m} + 2^{m/2} - 1$ over $GF(2^s)$ with $s = 2m + 1$ and $m$ even  \\ \hline
    Welch & $2^m + 3$ over $GF(2^s)$ with $s = 2m + 1$  \\ \hline
    \end{tabular}
\end{frame}

\begin{frame}
	\frametitle{What are our options for $GF(2^{16})$?}
	\begin{align*}
	d \in \{3, 1023, 63, 255, 15, 16383, \mathbf{65534}, 4095\}
	\end{align*}
	\begin{itemize}
		\item No Kasami, Gold, Welch, and Niho exponents exist for $n = 16$
		\item We need only study the \emph{inverse} and \emph{Dobbertin} exponents
	\end{itemize}

	\medskip

	\textbf{Action}: Complete security analysis by Monday, 6/3
\end{frame}

\begin{frame}
	\frametitle{Affine transformation update}
	\begin{itemize}
		\item Boolean function-related properties are unaffected by affine transformations 
		\item Need to select one with optimal algebraic complexity
		\begin{itemize}
			\item $S = A \circ P$ (maximum algebraic complexity is $n + 1$ for $GF(2^n)$) - efficient
			\item $S = A \circ P \circ A$ (maximum algebraic complexity is $2^n - 1$ for $GF(2^n)$) - inefficient
		\end{itemize}
	\end{itemize}
\end{frame}

% list others, and show dobbertins and list those that might be good/easy to implement

\begin{frame}
	\frametitle{Action Items}
	\begin{itemize}
		\item Quantified security results for all Dobbertin mappings and the inverse mapping
		\item Magma code to count the number of gates needed for multiplicative inverse calculation
		\item Continued writing for thesis (draft of all relevant chapters by Monday)
	\end{itemize}
	\begin{center}
		Next meeting: \textbf{6/3/13}
	\end{center}
\end{frame}

\end{document}
