\documentclass[handout]{beamer}
\usepackage{beamerthemesplit}
\usepackage{pgfpages}
\usepackage{verbatim}
\usepackage{amsmath, amssymb, graphics, setspace}

\usepackage{tikz}
\usetikzlibrary{arrows,%
                shapes,positioning}

\tikzstyle{vertex}=[circle,fill=black!25,minimum size=20pt,inner sep=0pt]
\tikzstyle{selected vertex} = [vertex, fill=red!24]
\tikzstyle{edge} = [draw,thick,-]
\tikzstyle{weight} = [font=\small]
\tikzstyle{selected edge} = [draw,line width=5pt,-,red!50]
\tikzstyle{ignored edge} = [draw,line width=5pt,-,black!20]

\newcommand{\field}[1]{\mathbb{#1}} %requires amsfonts

\usetheme{Antibes}
\usecolortheme{beaver}
\title[Composite Field Decomposition for Higher Order S-boxes]{Thesis Progress Report}

\usepackage{mathptmx}
\usepackage[scaled=.90]{helvet}
\usepackage{courier}
\usepackage[T1]{fontenc}

%\pgfpagesuselayout{4 on 1}[letterpaper,border shrink=5mm]

\institute[RIT]{}
\date{\today}
%\subtitle{}
\author{Christopher A. Wood}
%\institute[]{}
\date{\today}

\begin{document}

%%%%%
%%
%% Resource link: http://www.math-linux.com/spip.php?article77
%%
%%%%

\begin{frame}
	\titlepage
\end{frame}

\begin{frame}
	\frametitle{Agenda}
	\tableofcontents
\end{frame}

\section{Composite Field Mappings}
\begin{frame}
	\frametitle{Isomorphic Function Generation - Primitive Irreducible Polynomials}
	The isomorphic mappings are constructed as follows (assume we're constructing a function for mapping $GF(2^{nm}) \to GF((2^n)^m)$):
	\begin{itemize}
		\item Find two generators $\alpha$ and $\beta$ ($\alpha \in GF(2^{nm})$ and $\beta \in GF((2^n)^m)$), where $\alpha$ and $\beta$ are roots of the same \emph{primitive} irreducible polynomial.
		\item Map $\alpha^k$ to $\beta^k$ for $1 \leq k \leq 2^{nm}$.
		\item Multiplication and addition homomorphism is guaranteed. 
		\begin{itemize}
			\item $\alpha^i \times \alpha^j = \alpha^{i + j} = \beta^{i + j} = \beta^i \times \beta^j$
			\item $\alpha^t = \alpha^i + \alpha^j \to \beta^t = \beta^j + \beta^j$
		\end{itemize}
	\end{itemize}
\end{frame}

\begin{frame}
	\frametitle{Isomorphic Function Generation - Irreducible Polynomials}
	\begin{itemize}
		\item Find two generators $\alpha$ and $\beta$ ($\alpha \in GF(2^{nm})$ and $\beta \in GF((2^n)^m)$).
		\item Map $\alpha^k$ to $\beta^k$ for $1 \leq k \leq 2^{nm}$ (multiplication homomorphism holds)
		\item For all $0 \leq i \leq 2^{nm} - 1$ check to see if $\alpha^i + 1 \to \beta^i + 1$.
		\item Multiplication and addition homomorphism is now guaranteed.
		\begin{itemize}
			\item $\alpha^i \times \alpha^j = \alpha^{i + j} = \beta^{i + j} = \beta^i \times \beta^j$
			\item $\alpha^t = \alpha^i + \alpha^j = \alpha^i \times (1 + \alpha^{j - i}) \to \beta^t = \beta^i \times (1 + \beta^{j - i}) = \beta^i + \beta^j$
		\end{itemize}
	\end{itemize}
\end{frame}

\begin{frame}
	\frametitle{Matrix Transformation \textbf{T}}
	The matrix \textbf{T} can be generated with the following algorithm.
	\begin{itemize}
		\item Let $\beta$ be a generator of $GF((2^n)^m)$ such that $\alpha^i \in GF(2^{nm})$ is mapped to $\beta^i$ for all $0 \leq i \leq 2^{nm} - 1$ ($\alpha$ forms a basis of $GF((2^n)^m)$).
		\item Compute $\alpha^0, \alpha^1, \alpha^2,\dots,\alpha^{nm - 1}$.
		\item Define the columns of \textbf{T} as the transpose of each $nm$-dimensional bit vector of these powers:
\begin{align*}
		\textbf{T} =  \left[ \begin{array}{cccc}
(\alpha^{nm - 1})^T & \dots & (\alpha^{1})^T (\alpha^{0})^T \end{array} \right]
\end{align*}
	\end{itemize}
\end{frame}

\begin{frame}
\frametitle{An Example}
\begin{itemize}
	\item $\alpha = x$ and $\beta = xy$
	\item $(x^7 + x^6 + x^5 + x^2 + x + 1) \to [(x^3 + x^2 + x + 1)y + (x^3 + x^2 + 1)]$
	\item $\textbf{T} =  \left[ \begin{array}{cccc}
(xy^{nm - 1})^T & \dots & (xy^{1})^T (xy^{0})^T \end{array} \right]$
\end{itemize}

\begin{align*}
\left( \begin{array}{cccccccc}
0 & 0 & 1 & 0 & 0 & 0 & 1 & 0 \\
0 & 1 & 1 & 0 & 0 & 1 & 0 & 0 \\
0 & 0 & 0 & 1 & 1 & 0 & 1 & 0 \\
1 & 0 & 0 & 1 & 0 & 0 & 0 & 0 \\
1 & 1 & 1 & 0 & 1 & 0 & 0 & 0 \\
0 & 1 & 1 & 1 & 0 & 1 & 0 & 0 \\
1 & 0 & 1 & 0 & 1 & 0 & 0 & 0 \\
0 & 1 & 0 & 1 & 0 & 1 & 0 & 1 \\ \end{array} \right) \left( \begin{array}{c}
1 \\
1 \\
1 \\ 
0 \\
0 \\ 
1 \\
1 \\ 
1 \end{array} \right) = \left( \begin{array}{c}
1 \\
1 \\
1 \\ 
1 \\
1 \\ 
1 \\
0 \\ 
1 \end{array} \right)
\end{align*}
\end{frame}

\begin{frame}
\frametitle{Another Example}
\begin{itemize}
	\item $\alpha = x$ and $\beta = xy$ (same homomorphic mapping)
	\item $(x^6) \to [(x^2)y + (x^3 + x^2 + 1)]$
	\item $\textbf{T} =  \left[ \begin{array}{cccc}
(xy^{nm - 1})^T & \dots & (xy^{1})^T (xy^{0})^T \end{array} \right]$
\end{itemize}

\begin{align*}
\left( \begin{array}{cccccccc}
0 & 0 & 1 & 0 & 0 & 0 & 1 & 0 \\
0 & 1 & 1 & 0 & 0 & 1 & 0 & 0 \\
0 & 0 & 0 & 1 & 1 & 0 & 1 & 0 \\
1 & 0 & 0 & 1 & 0 & 0 & 0 & 0 \\
1 & 1 & 1 & 0 & 1 & 0 & 0 & 0 \\
0 & 1 & 1 & 1 & 0 & 1 & 0 & 0 \\
1 & 0 & 1 & 0 & 1 & 0 & 0 & 0 \\
0 & 1 & 0 & 1 & 0 & 1 & 0 & 1 \\ \end{array} \right) \left( \begin{array}{c}
0 \\
1 \\
0 \\ 
0 \\
0 \\ 
0 \\
0 \\ 
0 \end{array} \right) = \left( \begin{array}{c}
0 \\
1 \\
0 \\ 
0 \\
1 \\ 
1 \\
0 \\ 
1 \end{array} \right)
\end{align*}
\end{frame}

\section{Tower Field Decompositions}
\begin{frame}
	\frametitle{Different Tower Field Decompositions}
	\begin{align*}
	(1) GF(2^{16}) & \to GF((2^8)^2) \\ 
	(2) GF(2^{16}) & \to GF((2^4)^4) \\ 
	(3) GF(2^{16}) & \to GF((2^2)^8) \\
	\end{align*}
	We need only study these decompositions - optimal tower-field decompositions for smaller fields are in the literature.
\end{frame}

\begin{frame}
	\frametitle{Multiplicative Inverse Calculations}
	The derivation gets messy very quick...
	\begin{align*}
	(b*x^3+c*x^2+d*x+e)*(f*x^3+g*x^2+h*x+i)\text{=}\\
	k*(x^4+A*x^3+B*x^2+C*x+D)+1
	\end{align*}
	
	\begin{align*}
	f \to \frac{1}{x^3 \left(e+d x+c x^2+b x^3\right)}\left(1-e i+D k-e h x-d i x+C k x\right) \\ \left(-e g x^2-d h x^2-c i x^2+B k x^2-d g x^3-c
h x^3-b i x^3+A k x^3-c g x^4\right) \\ \left(-b h x^4+k x^4-b g x^5\right)
	\end{align*}
	\begin{center}
		Are there easier ways to calculate the inverse?
	\end{center}


\end{frame}

\begin{frame}
	\frametitle{Finding Capable Polynomials}
	\begin{itemize}
		\item Primitive polynomials always work for the mapping
		\begin{itemize}
			\item Let $\alpha$ be a primitive root of the field $F_2[x]/P(x)$ and $P(\alpha) = 0$
			\item $P(x)$ is therefore a \emph{primitive polynomial}
			\item Powers $\alpha$ (which are linearly independent) can be used to form a standard basis
		\end{itemize} 
		\item This didn't seem to work with non-primitive polynomials (i.e. $(x^8 + x^4 + x^3 + x + 1)$).
		\begin{itemize}
			\item \textbf{Question}: Why not? Group homomorphism between the elements holds...
		\end{itemize}
	\end{itemize}
\end{frame}

\begin{frame}
	\frametitle{Choosing the Right Irreducible Polynomials}
	\begin{itemize}
		\item Exhausively search for all (primitive) irreducible polynomials
		\item For each polynomial $P(x)$, generate the transformation matrix and 
		estimate the complexity of the multiplicative inverse calculation
		\begin{itemize}
			\item The polynomials $P(x)$, $P(y)$ and $Q(y)$ determine the complexity of this mathematication operation.
			\item $\alpha^{-1} = (bx + c)^{-1} = b(b^2B + bcA + c^2)^{-1} + (c + bA)(b^2B + bcA + c^2)^{-1}$
		\end{itemize}
		\item Choose the polynomial $P(x)$ that yields the lowest ``cost'' 
	\end{itemize}
\end{frame}

\begin{frame}
	\frametitle{Choosing the Right Transform}
	\begin{itemize}
		\item Let \textbf{T}$^*$ be the optimal transformation matrix in the set of transformations $\mathcal{T}$.
		\item The ``cost'' of transforms is the number of 1s in the matrix \textbf{T}$^*$. 
		\item The ``cost'' of the inverse is dependent on the polynomial selection.
	\end{itemize}
	\begin{align*}
	T^* = \min_{T_i \in \mathcal{T}}\{C(transform) + C(inverse) + C(invTransform)\}
	\end{align*}
\end{frame}

\begin{frame}
	\frametitle{Exhaustively Searching All S-boxes}
	\begin{itemize}
		\item Loop over invertible binary matrices and all constants for affine transformation
		\item For each valid mapping, measure the cryptographic strength using the Boolean function
		analysis software
		\item Pick the one with the best properties
	\end{itemize}
\end{frame}

\begin{frame}
	\frametitle{An Interesting Case}
	\begin{itemize}
		\item Nyberg's Power Mapping: $F(x) = x^{2^{k} + 1}$
		\item These functions are $2$-differentially unfiform with a $\mathcal{N}_l$ equal to precisely $2^{n-1} - 2^{\frac{n-1}{2}}$.
		\begin{itemize}
			\item That's better than the inverse mapping $F(x) = x^{-1}$
		\end{itemize}
		\item In a normal basis, this reduces to squaring (which is free) and multiplications
		\item For hardware, does this yield a more efficient \textbf{and} more secure mapping for $16$-bit S-boxes?
	\end{itemize}
\end{frame}

% \begin{frame}
% 	\frametitle{Step 2: Maiorana-McFarland Construction of Bent Functions}
% 	\begin{itemize}
% 		\item Bent functions have nonlinearity $2^{n-1} - 2^{\frac{n}{2} - 1}$.
% 		\item 
% 	\end{itemize}
% \end{frame}

% \begin{frame}
% 	\frametitle{Maiorana-McFarland}
% 	TODO: describe the technique here, describe an algorithm here...
% \end{frame}

\begin{frame}
	\frametitle{Combinational Implementations - XOR}
\begin{table}
	\begin{tabular}{|l|l|l|}
		\hline
		$X_1$ & $X_2$ & $Y$ \\ \hline
		0 & 0 & 0 \\ 
		0 & 1 & 1 \\
		1 & 0 & 1 \\ 
		1 & 1 & 0 \\
		\hline
	\end{tabular}
\end{table}

\begin{center}
	{\tt Y <= (NOT(X1) AND (X2)) OR (X1 AND NOT(X2))}
\end{center}

\begin{itemize}
	\item Walk the truth table output $Y$ and insert the appropriate literal assignments into the DNF formula
	\item I have the code to accomplish this task... 
\end{itemize}
\end{frame}

\begin{frame}
	\frametitle{Research Goal}
	What is the resource consumption and security tradeoff for Boolean function implementations of S-boxes
	versus those based on mathematical operations?
\end{frame}

% TODO: fill out sometime later...
\begin{frame}
	\frametitle{What's next?}
	\begin{center}
		Action Items
	\end{center}	
	\begin{itemize}
		\item Generate the list of all transformation matrices \textbf{T}
		\item List of all (primitive) irreducible polynomials up to degree $16$
		\item Exhaustively generate all S-boxes based on the inverse mapping $F(x) = x^{-1}$
		\item Thesis chapter on composite field arithmetic decomposition and inverse derivations
		using composite fields
	\end{itemize}
	\begin{center}
		Next meeting: \textbf{4/15/13}
	\end{center}
\end{frame}

\end{document}
