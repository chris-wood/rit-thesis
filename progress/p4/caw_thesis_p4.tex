\documentclass[handout]{beamer}
\usepackage{beamerthemesplit}
\usepackage{pgfpages}
\usepackage{verbatim}
\usepackage{amsmath, amssymb, graphics, setspace}

\usepackage{algorithm}
\usepackage{algorithmicx}
\usepackage{algpseudocode}


\newcommand{\field}[1]{\mathbb{#1}} %requires amsfonts

\usetheme{Antibes}
\usecolortheme{beaver}
\title[Composite Field Decomposition for Higher Order S-boxes]{Thesis Progress Report}

\usepackage{mathptmx}
\usepackage[scaled=.90]{helvet}
\usepackage{courier}
\usepackage[T1]{fontenc}

%\pgfpagesuselayout{4 on 1}[letterpaper,border shrink=5mm]

\institute[RIT]{}
\date{\today}
%\subtitle{}
\author{Christopher A. Wood}
%\institute[]{}
\date{\today}

\begin{document}

%%%%%
%%
%% Resource link: http://www.math-linux.com/spip.php?article77
%%
%%%%

\begin{frame}
	\titlepage
\end{frame}

\begin{frame}
	\frametitle{Agenda}
	\tableofcontents
\end{frame}

% (irr and primitive) poly generation algorithms
% 16-bit software and hardware for inverse
% sw implementation and performance measurements...
% boolean function analysis code and example...
% polynomial generation code...

\section{Generating Polynomials}
\begin{frame}[fragile]
	\frametitle{Testing for Irreducibility}
\begin{algorithm}[H]  %[htb]
\caption{Testing for polynomial irreducibility} 
\begin{algorithmic}[1]
\Require{A prime $p$ and \emph{monic} polynomial $f(x)$.}
\Ensure{YES if $f(x)$ is irreducible, NO otherwise.}
\State $g(x) \gets x$ \Comment{The smallest possible factor}
\For{$i = 1 \to \lfloor m/2 \rfloor$}
	\State $g(x) \gets g(x)^p \mod f(x)$
	\State $d(x) \gets \gcd\{f(x), g(x) - x\}$
	\If{$d(x) \not= 1$}
		\State NO
	\EndIf 
\EndFor
\Return YES
\end{algorithmic}
\end{algorithm}
\end{frame}

\begin{frame}[fragile]
	\frametitle{Order-Based Test for Irreducibility}
\begin{algorithm}[H] 
\caption{Testing for polynomial irreducibility} 
\begin{algorithmic}[1]
\Require{An integer $k$ and \emph{monic} polynomial $f(x)$.}
\Ensure{YES if $f(x)$ is irreducible, NO otherwise.}
\For{$u(x) = x \to x^{k-1} + x^{k-2} + \dotsb + x + 1$}
	\If{$u(x)$ is a generator of $GF(2^k)$ defined by $f(x)$}
		\Return YES
	\EndIf
\EndFor
\Return NO
\end{algorithmic}
\end{algorithm}
\end{frame}

\begin{frame}[fragile]
	\frametitle{Extending to Primitive Polynomials}
	\begin{itemize}
		\item A monic irreducible polynomial $f(x)$ is primitive in $GF(2^k)$ if its order is equal to $2^k - 1$.
		\item Alternatively, a monic irreducible polynomial is primitive if $x$ is a generator of $GF(2^k)$ defined by $f(x)$.
		\begin{itemize}
			\item Check to see that $x$ is in the list of generators.
		\end{itemize}
		\item Verify the correctness by counting:
		\begin{itemize}
			\item Primitive polynomial count: $a_q(n) = \frac{\phi(q^n - 1)}{n}$ 
			\item Irreducible polynomial count: $L_q(n) = \frac{1}{n}\sum_{d | n}\mu(n/d)q^d$
		\end{itemize}
	\end{itemize}
\end{frame}

\begin{frame}[fragile]
	\frametitle{Extending the algorithms to composite fields}
\begin{algorithm}[H] 
\caption{Testing for polynomial irreducibility} 
\begin{algorithmic}[1]
\Require{}
\Require{Positive integers $n$ and $m$, an irreducible polynomial $p(x)$ in $GF(2^n)$, and candidate polynomial $q(y)$ in $GF((2^n)^m)$.}
\Ensure{YES if $f(x)$ is irreducible, NO otherwise.}
\For{$u(y) = y \to y^{m-1} + (x^{n-1} + \dotsb + x + 1)y^{m-2} + \dotsb + (x^{n-1} + \dotsb + x + 1)y + (x^{n-1} + \dotsb + x + 1)$}
	\If{$u(y)$ is a generator of $GF((2^n)^m)$ defined by $p(x)$ and $q(y)$}
		\Return YES
	\EndIf
\EndFor
\Return NO
\end{algorithmic}
\end{algorithm}

\begin{center}
A similar test for primitivity holds here (check to see if $y$ is a generator of $GF((2^n)^m)$)
\end{center}
\end{frame}

\section{Finite Field Arithmetic in Software}
\begin{frame}
	\frametitle{Multiplicative Inverse Calculations}
	\begin{itemize}
		\item $GF(2^{16})$ is relatively small compared to other fields (e.g. ECC)
		\item Yet, we should strive for two types of software implementations under two assumptions:
		\begin{enumerate}
			\item Constant time computations are required  
			\item No memory constraints and no possibility for side-channel attacks
		\end{enumerate}
	\end{itemize}
\end{frame}

\begin{frame}
	\frametitle{Software Review}
	\begin{itemize}
		\item Composite field library
		\item Polynomial generation
		\item Isomorphic function generation 
		\item 16-bit inverse calculation
		\item Boolean function analysis
	\end{itemize}
\end{frame}

\section{Measuring Boolean Functions}
\begin{frame}
	\frametitle{Cryptographically Significant S-box Metrics}
	\begin{itemize}
		\item Linear and differential approximations (tables will be large)
		\item Boolean differential uniformity
		\item Boolean function nonlinearity
		\item Algebraic immunity
		\item Correlation immunity
		\item Resiliency
	\end{itemize}
\end{frame}

\begin{frame}
	\frametitle{Questions to Answer}
	\begin{itemize}
		\item How many irreducible and primitive polynomials exist for extension fields?
		\item Can we optimize the isomorphic function generation algorithm?
		\item Which cryptographic properties are most important? Should nonlinearity be more critical than correlation immunity?
		\item How can Boolean functions be efficiently implemented in software? They generally have no algebraic constructions. Are large sequences of bitwise operations or LUTS the only ways?
		\item Without using an affine transformation, how can we make the S-box expression \emph{algebraically complex}?
	\end{itemize}
\end{frame}

\begin{frame}
	\frametitle{Action Items}
	\begin{itemize}
		\item Literature survey of software optimizations for Galois field arithmetic and AES-specific operations
		\item Write the introduction chapter for thesis (S-box motivation, attacks, etc)
		\item Exhaustive list of all polynomials $P(x)$, $Q(y)$, and $R(z)$ and the 
		corresponding list of all transformation matrices (using OSG!)
		\item Composite-field implementation of 16-bit inverse in software and hardware
		\item Preliminary Boolean function construction code
	\end{itemize}
	\begin{center}
		Next meeting: \textbf{4/29/13}
	\end{center}
\end{frame}

\end{document}
