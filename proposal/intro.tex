\section{Problem Statement}
The cryptographic strength of symmetric-key cryptosystems is traditionally based on 
Claude Shannon's properties of confusion and diffusion \cite{Shannon}. 
Confusion can be defined as the complexity of the relationship between the secret key and 
ciphertext, and diffusion can be defined as the degree to which the influence of 
a single input plaintext bit is spread throughout the resulting ciphertext.
Substitution-permutation networks (SPNs) are natural constructions for symmetric-key
cryptosystems that realize confusion and diffusion through substitution and
permutation operations, respectively \cite{Stinson}. 

As the only nonlinear operation in SPNs, the substitution step, 
more commonly referred to as an S(ubstitution)-box, is critically important
in the construction of cryptographically strong block ciphers that are resilient to
common attacks, including linear and differential cryptanalysis, as well as algebraic 
attacks. Furthermore, as these cryptanalysis efforts have evolved over the past 
few decades, and with the selection of the Advanced Encryption 
Standard (AES) symmetric-key block cipher, the construction 
of cryptographically strong S-boxes with efficient
hardware and software implementations in these 
cryptosystems has become a topic of critical research. 

An S-box mapping can be defined as a function $F : GF(2^n) \to GF(2^m)$. To measure
the cryptographic strength of these functions it is common to represent them as vectorial Boolean functions,
where $F(x) = (f_1(x), \dots, f_m(x))$ and $f_i : GF(2^n) \to GF(2)$, for all $1 \leq i \leq m$. Such a representation
enables one to measure its nonlinearity (i.e. measure of distance from affine functions), algebraic
immunity (i.e. difficulty measure of annihilator-based algebraic attacks), resiliency (i.e. balancedness 
and correlation immunity between input and output of the function), and differential uniformity (i.e.
difficulty measure of differential cryptanalysis). 

Practical S-boxes must also be efficiently computable. For this reason, they are typically defined as functions
over finite fields $GF(2^k)$, where $2 | k$. For example, the Rijndael S-box 
is defined as the function $F : GF(2^8) \to GF(2^8)$, $F(a) = Ba^{-1} \oplus c$, where $B$ and
$c$ are a constant $8 \times 8$ matrix and $8$-dimensional vector over $GF(2)$ \cite{Daemen02-1}. If we represent an
input element $a$ and output element $b$ as $a_7a_6a_5a_4a_3a_2a_1a_0$ and 
$b_7b_6b_5b_4b_3b_2b_1b_0$, respectively, where $a_i,b_i \in GF(2)$ for for all 
$0 \leq i \leq 7$, then the S-box $F$ can be defined as:
\begin{align*}
\begin{bmatrix} b_7 \\ b_6 \\ b_5 \\ b_4 \\ b_3 \\ b_2 \\ b_1 \\ b_0 \end{bmatrix} = 
\begin{bmatrix} 
1 & 1 & 1 & 1 & 1 & 0 & 0 & 0 \\
1 & 1 & 1 & 1 & 1 & 0 & 0 & 0 \\
1 & 1 & 1 & 1 & 1 & 0 & 0 & 0 \\
1 & 1 & 1 & 1 & 1 & 0 & 0 & 0 \\
1 & 1 & 1 & 1 & 1 & 0 & 0 & 0 \\
1 & 1 & 1 & 1 & 1 & 0 & 0 & 0 \\
1 & 1 & 1 & 1 & 1 & 0 & 0 & 0 \\ 
1 & 1 & 1 & 1 & 1 & 0 & 0 & 0 \end{bmatrix} \begin{bmatrix} a_7 \\ a_6 \\ a_5 \\ a_4 \\ a_3 \\ a_2 \\ a_1 \\ a_0 \end{bmatrix}^{-1} \oplus
\begin{bmatrix}
c_7 \\ c_6 \\ c_5 \\ c_4 \\ c_3 \\ c_2 \\ c_1 \\ c_0
\end{bmatrix}
\end{align*}
With only $2^8$ elements in $GF(2^8)$, it is possible to tabulate and store
all values $b = F(a)$ for all $a \in GF(2^8)$. With such a lookup table (LUT) data structure, the SubByte
step in the Rijndael algorithm can be computed with a single index into this table. 
There are however some critical flaws with this approach in practical implementations
of the Rijndael algorithm.

First, this implementation strategy is susceptible to cache-timing 
side-channel attacks on systems where the CPU cache size is too small to store the 
entire Rijndael state and S-box LUT. On such systems, it is possible to conduct 
a chosen-plaintext attack requiring anywhere from 
$2^{13}$ to $2^{20}$ samples to perform a full key recovery \cite{Bonneau06-1}. The number of
samples required for this attack is highly dependent on the system properties
and the accuracy of the timing mechanisms available to the attacker. These attacks
are based on timing anomalies that result from cache-collisions (cache hits) 
that occur when performing table lookups at runtime. More specifically, it is possible
to relate bits of the input (or output) to bits of the key that are satisfied
when timing deviations occur. By exploiting this relationship it is possible
to deduce parts of the key until the it can eventually be fully recovered 
or the remaining key space is small enough to perform an exhaustive search. 

Another problem with LUT-based implementations of the S-box is that they require
$2$Kb of memory. In resource constrained systems, 
such as VLSI circuits and embedded platforms, such memory resources
are not readily available. 

Clearly, constant-time S-box mappings with small memory footprints, which
can be achieved at runtime with online calculations in software or combinational logic 
in hardware, are ideal for secure and efficient implementations
of the block ciphers with S-boxes similar to Rijndael. 
Hardware optimization efforts are traditionally focused on
the multiplicative inverse calculation in the S-box, since this is the 
most computationally expensive procedure. 
To date, the most effective technique for minimizing the circuitry required
for the multiplicative inverse calculation is to use composite field
arithmetic to replace the inverse calculation of $a \in GF(2^8)$ with
a series of multiplication, squaring, addition, and an inverse operation
on elements $b,c \in GF(2^4)$. Similar decomposition strategies
can be implemented and optimized in software to yield efficient and 
near constant-time S-box calculations in memory-constrained systems.

The problem of optimizing hardware and software implementations of S-boxes has been 
studied in the literature for more than a decade \cite{Satoh01-1, Mentens05-1, Canright05-1, Boyar12-1}.
Furthermore, with the selection of the Advanced Encryption Standard,
all of this research has been focused on 8-bit S-boxes (i.e. $F : GF(2^8) \to GF(2^8)$).
To our knowledge, there has not been any work that studies the strength and
implementation efficiency (in terms of software throughput and hardware area) of higher order
S-boxes. Therefore, in this thesis, we break away from tradition and focus our 
attention on $16$-bit S-boxes. The problem is to study these S-boxes
in the context of Boolean functions to determine their strength, and for each ideal candidate
S-box, attempt to find optimized hardware and software implementations through 
composite field arithmetic with polynomial bases. Time permitting, we may also explore the 
efficacy of normal and mixed basis decomposition strategies for optimizing the S-box calculations.

