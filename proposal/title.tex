\begin{titlepage}

\thispagestyle{empty}
\begin{center}
\vspace*{3em}

{\LARGE \bf Large Substitution Boxes \\with Efficient Combinational Implementations}
\vspace*{2em}

{\Large Christopher A. Wood}
\vspace*{.5em}

{\small Department of Computer Science\\
  Rochester Institute of Technology\\
  Rochester, NY 14623 USA\\
  {\tt caw4567@rit.edu }}

\vspace*{1em}

{\large \it Master's Thesis Proposal}

\vspace*{1em}

{\large
\renewcommand{\arraystretch}{2}
\begin{tabular}{ l c r }
  Chair: & {\large Professor Stanis{\l}aw P. Radziszowski} & {\tt spr@cs.rit.edu}\\[1ex]
  \multicolumn{3}{ c }{\noindent\rule{8cm}{0.4pt}}\\
  Reader: & Professor Marcin Lukowiak & {\tt mxleec@rit.edu}\\[1ex]
  \multicolumn{3}{ c }{\noindent\rule{8cm}{0.4pt}}\\
  Observer: & Professor Alan Kaminsky & {\tt ark@cs.rit.com}\\[1ex]
  \multicolumn{3}{ c }{\noindent\rule{8cm}{0.4pt}}\\
  Observer: & Dr. Michael Kurdziel & {\tt mkurdzie@harris.com}\\[1ex]
  \multicolumn{3}{ c }{\noindent\rule{8cm}{0.4pt}}
\end{tabular}
}

{\large \today}

\end{center}

\vspace*{1em}
%\vfill

\begin{abstract}
The cryptographic strength of S-boxes comes at the price of computational efficiency.
In resource constrained systems, such as VLSI circuits and embedded platforms, traditional LUT-based
implementations are not feasible. Cryptographers and engineers have worked for many years to find a balance between the strength
and efficiency of S-boxes. Furthermore, with the selection of the Advanced Encryption Standard,
the majority of this research has focused on $8$-bit S-boxes. 
In this thesis, we focus our attention on $16$-bit S-boxes. We propose to study these S-boxes
in the context of Boolean functions to determine their strength, and for each ideal candidate
S-box, attempt to find optimized hardware and software implementations.
\end{abstract}

\end{titlepage}
