\section{Proposed Work}
The first part of our research will focus on the cryptographic strength of 
various S-box definitions using analysis methods for Boolean functions. Clearly, exhaustively searching
all $GF(2^{2^{16}})$ S-box definitions is infeasible, so our analysis will be constrained
to S-boxes defined by differentially uniform mappings over Galois fields, such as the
inverse mapping $F(x) = x^{-1}, x \in GF(2^{16})$, and built 
using the Maiorana-McFarland Boolean function construction technique. Time
permitting, we may also explore Boolean function construction techniques targeted towards specific
cryptographic properties, such as resiliency and algebraic immunity. 
The metrics collected for this part of the research include the nonlinearity,
algebraic immunity, resiliency, and differential uniformity of each S-box candidate. 
Some of these metrics can be gathered using third-party software tools like 
SAGE and Mathematica, while others will require custom software to measure. Such
software would be part of the deliverables for this thesis, and it will be used
to weigh the strength and weaknesses of various S-box definitions. 
 
The second thread of our research will be to implement candidate S-boxes in FPGA hardware and generate 
similar equivalent ASIC models. To facilitate the selection of candidate S-boxes, 
we will study the gate-level complexity of ideal S-boxes using various
representations defined by different irreducible polynomials. These implementation techniques will explore the composite field
decomposition techniques using polynomial bases. Time permitting, we will also explore the optimizations
that are possible with normal basis representations of the S-boxes. While mixing these two representations
will likely lead to an improved solution, such work is outside the scope of this thesis. A major outcome
of this thread of research is to determine the implementation differences between FPGA and ASIC platforms
in terms of hardware resource and power consumption. Composite field decompositions do not currently benefit
on FPGA platforms due to the small size of logic required for the functions and the larger LUT blocks. 
It is unclear as to whether or not this problem will be subverted with 16-bit S-boxes, and so we hope 
to address this issue. 

Software versions of these functions will also be implemented to measure the throughput performance
on low-end platforms (e.g. 32-bit PowerPC hardcore processors).
The metrics collected for this part of the research include hardware area (i.e.
FPGA slices and total synthesized logic gates) and throughput (cycles per byte), as well
as the software memory footprint for S-box functions and their throughput (cycles per byte).
These software implementations will be compared against LUT-based implementations in terms of 
memory usage and throughput. 