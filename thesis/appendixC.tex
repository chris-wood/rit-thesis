% \begin{minted}[fontsize=\tiny,frame=single]{c}
%   \input{../../src/GaloisLibrary/galois.c}
% \end{minted}

\section{Galois Field Composite Arithmetic Library}
In this section we provide a brief tutorial on how to use the ``toy'' Galois field composite arithmetic library.
As a whole, it only supports basic arithmetic for $GF(p)$ and $GF(q^n)$, $q = p^m$. Even with this limitation,
it can be quite useful for those trying to understand the mathematical concepts of composite fields
and the corresponding arithmetic, as seemingly simple as it may be. 

Consider the field $GF(2^4)$ defined by the irreducible polynomial $p(x) = x^4 + x + 1$. Let $\alpha = x^3 + 1$ be an element in this field. We can
create this Galois field element and perform arithmetic in the field $GF(2^4)$ as follows:

\begin{minted}[fontsize=\tiny,frame=single]{python}
>>> from galois import *
>>> ip = GFElem([1,0,0,1,1]) 
>>> F = GF(2, 4, ip)
>>> print(F)
GF(2^4), P(x) = (1x^4 + 1x^1 + 1x^0)
>>> x = GFElem([1,0,0,1])
>>> print(x)
(1x^3 + 1x^0)
>>> xa = F.g_add(x, x)
>>> print(xa)
0
>>> xs = F.g_mult(x, x)
>>> print(xs)
(1x^3 + 1x^2 + 1x^0)
>>> print(F.g_mult(F.inverse(x), x))
(1x^0)
\end{minted} 

Creation of Galois field elements is as simple as providing a list of coefficients in the 
ground field. Also, even though the example works with binary fields only, the library 
can support any prime $p$ as the field characteristic. 

Now assume we wish to create the a degree-2 extension of this field, i.e. $GF((2^4)^2)$, 
using the irreducible polynomial $y^2 + y + (x^3 + x^2)$. Notice that the coefficients of
each term are elements of the field $GF(2^4)$, as per our earlier definition. Let 
$\beta = (x^2 + 1)y + (x^2 + x + 1)$ be an element in $GF((2^4)^2)$. We can create this field
and element to perform basic arithmetic as in the base field as follow.

\begin{minted}[fontsize=\tiny,frame=single]{python}
>>> eIp = GFExtensionElem([GFElem([1]), GFElem([1]), GFElem([1,1,0,0])]) 
>>> F2 = GFExtension(F, 2, eIp)
>>> print(F2)
2 degree extension of: GF(2^4)
>>> b = GFExtensionElem([GFElem([1,0,1]),GFElem([1,1,1])])
>>> print(b)
[(1x^2 + 1x^0)y^1 + (1x^2 + 1x^1 + 1x^0)y^0]
>>> ba = F2.g_add(b, b)
>>> print(ba)
0
>>> bs = F2.g_mult(b, b)
>>> print(bs)
[(1x^1)y^1 + (1x^3 + 1x^2 + 1x^0)y^0]
>>> bi = F2.power(b, 254) # Fermat's Little Theorem says this should be the inverse
>>> print(bi) 
[(1x^0)y^0]               # And so it is...
\end{minted}

The reader is welcome to browse the library source code and play with
additional features (e.g. finding generators for a field) at their own leisure.

\section{Circuit Minimization Programs}
Here we provide relevant snippets from the parallel implementations of logic minimization programs. The full versions will be available with the distribution of this work.

\subsection{Parallel Factorization Program (snippet)}
\begin{minted}[fontsize=\tiny,frame=single]{python}
public void compute() throws Exception
{
    ArrayList<Pair> pairs = new ArrayList<Pair>();
    for (int i = 0; i <= lastcol - 1; i++)
    {
        for (int j = i + 1; j <= lastcol; j++)
        {
            if (!(i < oldi && j != oldi && j != oldj && j < lastcol)) 
            {
                int[] coli = currMatrix.getColumn(i);
                int[] colj = currMatrix.getColumn(j);
                int wt = weight(AND(coli, colj));
                if (wt > 1) 
                {
                    pairs.add(new Pair(i, j));
                }
            }
        } 
        
    }
    if (pairs.size() > 0)
    {
        exhaustiveOptimize(pairs); // breadth first
    }
}

public void exhaustiveOptimize(ArrayList<Pair> pairs) throws Exception
{
    int wt = 0;
    int hmax = 0;
    HashSet<ParallelMatrixOptimize> tasks = new HashSet<ParallelMatrixOptimize>();

    for (Pair p : pairs)
    {
        int i = p.i;
        int j = p.j;
        int[] coli = currMatrix.getColumn(i);
        int[] colj = currMatrix.getColumn(j);
        int[] newcol = AND(coli, colj);
        Matrix newMatrix = new Matrix(currMatrix);
        newMatrix.appendColumn(newcol);
        for (int k = 0; k < n; k++)
        {
            if (newcol[k] == 1)
            {
                newMatrix.setEntry(k, i, 0);
                newMatrix.setEntry(k, j, 0);
            }
        }

        ParallelSolution sol = new ParallelSolution(newMatrix.getGateCount(), 
            newMatrix, i, j, n, lastcol + 1);
        tasks.add(new ParallelMatrixOptimize(sol));
    }

    invokeAll(tasks); // fork & join
    for (ParallelMatrixOptimize result : tasks)
    {
        if (result.solution.gOptimal < this.solution.gOptimal)
        {
            this.solution = result.solution;
        }
    }
}
\end{minted}

\subsection{Parallel Boyar-Peralta Program (snippet)}
\begin{minted}[fontsize=\tiny,frame=single]{python}
public static SLP parallelPeraltaOptimize(final Matrix m, final int r, final int c, final int tieBreaker) 
{
    p_xorCount = 0;
    solutionCircuit.reset();
    p_dist = null;
    p_slp = null;
    p_base = null;
    p_n = 0;
    p_newDist = null;
    p_optimalBases = null;
    p_optimalDistances = null;
    p_pairs = null;
    p_slp = new ArrayList<String>();

    // Create the initial base
    int[][] b = new int[c][c];
    for (int i = 0; i < c; i++)
    {
        for (int j = 0; j < c; j++)
        {
            if (i == j)
            {
                b[i][j] = 1;
            }
        }
    }
    p_base = new Matrix(b, c);
    
    // Initialize the distance array and then get the ball rolling
    p_dist = computeDistance(p_base, m);
    for (int f = 0; f < p_dist.length; f++)
    {
        if (p_dist[f] == 0)
        {
            int[] row = m.getRow(f);
            for (int cc = 0; cc < row.length; cc++)
            {
                if (row[cc] == 1)
                {
                    p_slp.add("y_" + f + " = " + "x_" + cc);
                }
            }
        }
    }

    p_newDist = new int[m.getDimension()];
    p_i = m.getLength();
    final int nVars = m.getLength();

    p_n = p_base.getDimension(); 
    p_d = sum(p_dist);
    p_optimalBase = new int[p_base.getLength()]; 
    p_optimalBases = new ArrayList<int[]>();
    p_optimalDistances = new ArrayList<int[]>();
    p_pairs = new ArrayList<Pair>();

    // Create the parallel team so everything isn't anonymous
    ParallelTeam team = new ParallelTeam();
    while (isZero(p_dist) == false)
    {
        team.execute(new ParallelRegion() 
        {
            public void run() throws Exception 
            {
                execute (0, p_n - 1, new IntegerForLoop()
                {
                    int t_d = p_d;
                    ArrayList<int[]> t_optimalBases = new ArrayList<int[]>();
                    ArrayList<int[]> t_optimalDistances = new ArrayList<int[]>();
                    ArrayList<Pair> t_pairs = new ArrayList<Pair>();

                    public IntegerSchedule schedule()
                    {
                        return IntegerSchedule.guided();
                    }

                    public void run (int first, int last) throws Exception
                    {
                        for (int t_ii = first; t_ii <= last; t_ii++)
                        {
                            for (int t_jj = t_ii + 1; t_jj < p_n; t_jj++)
                            {
                                if (t_ii != t_jj)
                                {
                                    // t_newDist = new int[m.getDimension()];

                                    int[] ssum = addMod(p_base.getRow(t_ii), p_base.getRow(t_jj), 2);
                                    if (!(p_base.containsRow(ssum)) && !isZero(ssum))
                                    {
                                        int[] t_newDist = optimizedComputeDistance(p_base, m, ssum, p_dist);
                                        int newDistSum = sum(t_newDist);
                                        if (newDistSum < t_d)
                                        {
                                            t_d = newDistSum;
                                            t_optimalBases = new ArrayList<int[]>();
                                            t_optimalDistances = new ArrayList<int[]>();
                                            t_pairs = new ArrayList<Pair>();
                                            t_optimalBases.add(ssum);
                                            t_optimalDistances.add(t_newDist);
                                            t_pairs.add(new Pair(t_ii, t_jj));
                                        }
                                        else if (newDistSum == t_d)
                                        {
                                            t_optimalBases.add(ssum);
                                            t_optimalDistances.add(t_newDist);
                                            t_pairs.add(new Pair(t_ii, t_jj));
                                        }
                                        ssum = null;
                                    }   
                                }
                            }
                        }
                    }

                    public void finish() throws Exception
                    {
                        if (t_d < p_d)
                        {
                            solutionCircuit.record(t_d, t_optimalBases, 
                              t_optimalDistances, t_pairs);
                        }
                    }
                });
            }
        });

        // Copy over the solution
        p_d = solutionCircuit.d;    
        p_optimalBases = solutionCircuit.optimalBases;
        p_optimalDistances = solutionCircuit.optimalDistances;
        p_pairs = solutionCircuit.pairs;

        // Resolve using norms
        int mi = 0;
        int mj = 0;
        double maxNorm = Double.MAX_VALUE;
        switch (tieBreaker)
        {
            // Omitted for brevity
        }

        // Generate & save the new base
        BasePair newBase = new BasePair(p_optimalBase, p_newDist, new Pair(mi, mj));
        p_base.appendRow(newBase.base);

        // Insert the new line into the program
        String c1 = newBase.p.i < nVars ? "x" : "t";
        String c2 = newBase.p.j < nVars ? "x" : "t";
        p_slp.add("t_" + p_i + " = " + c1 + newBase.p.i + " XOR " + c2 + newBase.p.j);
        p_xorCount++;
        for (int f = 0; f < m.getDimension(); f++)
        {
            if (areEqual(newBase.base, m.getRow(f)))
            {
                p_slp.add("y_" + f + " = " + "t_" + p_i);
            }
        }
        p_i++;

        // Update shared variables for the next round
        p_n = p_base.getDimension(); 
        p_d = sum(p_dist);
        p_optimalBase = null;
        p_optimalBase = new int[p_base.getLength()]; // was dimension
        p_optimalBases = null;
        p_optimalBases = new ArrayList<int[]>();
        p_optimalDistances = null;
        p_optimalDistances = new ArrayList<int[]>();
        p_pairs = null;
        p_pairs = new ArrayList<Pair>();
        for (int k = 0; k < p_newDist.length; k++)
        {
            p_dist[k] = p_newDist[k];
        }
    }

    return new SLP(p_slp, p_xorCount, 0); 
}
\end{minted}

\section{S-Box Gate Counting Program}
\subsection{gateCount.m (snippet)}
\inputminted[fontsize=\tiny,frame=single]{java}{../../src/CompositeFields/magma/gateCount_snippet.m}

\section{Multiplicative Inverse Calculation using Composite Field Arithmetic}
\subsection{galois.c}
\inputminted[fontsize=\tiny,frame=single]{c}{../../src/GaloisLibrary/galois.c}
\subsection{galois.h}
\inputminted[fontsize=\tiny,frame=single]{c}{../../src/GaloisLibrary/galois.h}

\section{Alternative AES S-Box Hardware Design}
The following VHDL code gives a complete hierarchical design for the forward function of our alternate AES S-box using the field polynomial $s(v) = v^8 + v^6 + v^5 + v^4 + v^2 + v + 1$. With little effort one may implement both the inverse and merged designs by inserting the appropriate basis change matrices and constant additions before and after the inversion operator.
\inputminted[fontsize=\tiny,frame=single]{vhdl}{../../src/Hardware/aes_alternative/forward_sbox/all.vhd}

\section{16-Bit S-box in C}
\inputminted[fontsize=\tiny,frame=single]{c}{../../src/GaloisLibrary/sbox16_unopt.c}
