\begin{titlepage}

\thispagestyle{empty}
\begin{center}
\vspace*{5em}

{\LARGE \bf Large Substitution Boxes \\with Efficient Combinational Implementations}
\vspace*{2em}

{\Large Christopher A. Wood}
\vspace*{.5em}

{\small Department of Computer Science\\
  Rochester Institute of Technology\\
  Rochester, NY 14623 USA\\
  {\tt caw4567@cs.rit.edu }}

\vspace*{3em}

{\large \it Masters Thesis}

\vspace*{3em}

{\large
\renewcommand{\arraystretch}{2}
\begin{tabular}{ l c r }
  Chair: & {\large Professor Stanis{\l}aw P. Radziszowski} & {\tt spr@cs.rit.edu}\\[1ex]
  \multicolumn{3}{ c }{\noindent\rule{8cm}{0.4pt}}\\
  Reader: & Professor Marcin Lukowiak & {\tt mxleec@rit.edu}\\[1ex]
  \multicolumn{3}{ c }{\noindent\rule{8cm}{0.4pt}}\\
  Observer: & Professor Alan Kaminsky & {\tt ark@cs.rit.edu}\\[1ex]
  \multicolumn{3}{ c }{\noindent\rule{8cm}{0.4pt}}\\
  Observer: & Dr. Michael Kurdziel & {\tt mkurdzie@harris.com}\\[1ex]
  \multicolumn{3}{ c }{\noindent\rule{8cm}{0.4pt}}
\end{tabular}
}

\vspace*{3em}

{\large \today}

\end{center}

\vfill

% \begin{abstract}
% Substitution-permutation network (SPN) designs are very popular constructions for 
% symmetric-key cryptographic primitives. The Advanced Encryption Standard is one 
% prime example of an SPN block cipher that adheres to this design strategy. The substitution
% step, often referred to as the S-box, is typically the only nonlinear component
% of such designs that help realize Claude Shannon's principle of confusion - increasing
% the complexity of the relationship between the secret key and the input plaintext. 
% As a result, much research has been devoted to improving the cryptographic strength and implementation
% efficiency of S-boxes so as to prohibit cryptanalysis attacks that exploit
% weak constructions and enable fast and area-efficient hardware implementations on 
% a variety of platforms. Most S-boxes in the literature are bijective functions that have
% $4$ or $8$ bit inputs. In this work, we explore the cryptographic properties and implementation
% options of $16$ bit S-boxes in hopes of stimulating a new perspective of 
% cryptographic research. We study these S-boxes in the context of Boolean functions 
% to determine their strength and present low-area combinational hardware implementations
% for each candidate S-box.
% \end{abstract}

\end{titlepage}
