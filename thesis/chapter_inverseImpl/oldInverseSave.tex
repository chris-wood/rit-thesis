
\subsection{Extended Euclidean Algorithm}
The Euclidean algorithm is perhaps one of the most widely known technique for computing
the greatest common divisor of two integers. The basic algorithm is shown in Algorithm \ref{alg:euclidean},
with a sample run for two arbitrary non-negative integers $a$ and $b$ shown below.

\begin{algorithm}[ht!] %[htb]
\caption{EuclideanAlgorithm($a, b$)} \label{alg:euclidean}
\begin{algorithmic}[1]
	\While {$b \not= 0$}
		\State $r \gets a \mod b$
		\State $a \gets b$
		\State $b \gets r$
	\EndWhile
	\State \Return $a$
\end{algorithmic}
\end{algorithm}

By definition, we know that $\gcd(a, b) = ax + by$ (B\`{e}zout's identiity) for some $x,y \in \mathbb{Z}$. 
It is easy to see that computing the values of $x$ and $y$ are encapsulated in the iterations
of the Euclidean algorithm, and so with a simple modification to this code we may compute integers $x$ and
$y$ that satisfy B\`{e}zout's identiity. This variation is known as the Extended Euclidean algorithm.
If we let $y = p(x)$, the irreducible polynomial for the field $GF(2^n)$, then 
$ax + by \equiv ax$, and so $a$ must be the inverse of $x$,
as desired. It is quite easy to see that this algorithm works for elements $\alpha \in GF(2^k)$ as well.
The Binary Extended Euclidean Algorithm, shown in Algorithm \ref{alg:binaryEEA} \cite{Guajardo06-1}, 
is a special variant of the EEA which uses more efficient divison operations in which the divisor 
is always a power of two (i.e. division equates to bit shifting), which leads to efficient implementations 
in software. However, since this requires sequential logic, we cannot use this technique
for combinational implementations.

\begin{algorithm}[ht!] %[htb]
\caption{BinaryGaloisFieldEEA($\alpha$)} \label{alg:binaryEEA}
\begin{algorithmic}[1]
	\State $b \gets 1$, $c \gets 0$, $u \gets \alpha$, $v \gets p(x)$
	\While{$x \mod u \equiv 0$}
		\State $u \gets u/x$
		\If{$x \mod b \equiv 0$}
			\State $b \gets b/x$
		\Else
			\State $b \gets (b + p(x)) / x$
		\EndIf
	\EndWhile
	\If {$u = 1$}
		\State \Return $b$
	\EndIf
	\If{$deg(u) < deg(v)$}
		\State $swap(u, v)$, $swap(b, c)$
	\EndIf
	\State $u \gets u + v$, $b \gets b + c$
	\State \textbf{goto} Step 2
\end{algorithmic}
\end{algorithm}

Unfortunately, the iterative nature of these algorithms means that they cannot be implemented
in combinational logic. For the purpose of deriving direct combinational implementations of the 
multiplicative inverse over $GF(2^n)$, we discard this as a viable option for hardware implementations
with small area footprints. 

\subsection{Fermat's Little Theorem}
Fermat's Little Theorem states that for any prime number $p$ and integer $a$, $a^p - a$ is
an integer multiple of $p$. In other words, $p | a^p -a$, or $a^p \equiv a \mod p$. Furthermore,
if $\gcd\{a,p\}=1$, then $a^{p-1} \equiv 1 \mod p$. The same holds true for elements of finite fields.
For every element $\alpha \in GF(2^k)$ defined by the irreducible polynomial $p(x)$, we 
have $\alpha^{-1} \equiv \alpha^{2^k - 1}\alpha^{-1} \equiv \alpha^{2^k - 2} \mod p(x)$.
Naively squaring and reducing the element $\alpha$ yields a time complexity of $\mathcal{O}(2^kM(n))$, 
where $M(n)$ is the complexity of the multiplication algorithm used. However, 
note that the element $\alpha^{2^k - 2}$ can be computed as follows:
\begin{align*}
\alpha^{2^{n} - 2} = \alpha^{2^{1}} \cdot \alpha^{2^{2}} \dotsb \alpha^{2^{n-1}}
\end{align*}
This requires only $n - 2$ multiplications and $n - 1$ squarings
(as was done by Wang et al. \cite{Wang85-1} and shown in Algorithm \ref{alg:wangInverse}), and therefore
has a time complexity of $\mathcal{O}(\log_2 k)$. Note that if $\alpha$ was 
represented in a normal basis, all squaring operations would reduce to cyclic 
shifts, which are free in hardware and very inexpensive in software.
\begin{algorithm}[t] %[htb]
\caption{Wang Inverse} \label{alg:wangInverse}
\begin{algorithmic}[1]
	\Require{$\alpha \in GF(2^k)$}
	\State $y \gets \alpha$
	\For{$k := 1 \to k - 2$}
		\State $z \gets y^2$
		\State $y \gets zx$ % \Comment{Multiplication in $GF(2^k)$}
	\EndFor
	\State $y \gets y^2$
	\State \Return $y$
\end{algorithmic}
\end{algorithm}