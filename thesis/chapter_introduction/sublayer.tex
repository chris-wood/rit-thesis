\section{Cryptanalysis Attacks} \label{sec:attacks}
The substitution layer plays a critical role in the security of block ciphers designed with a substitution layer for nonlinearity. Over the past several decades, many forms of cryptanalysis attacks have been devised, implemented, and tested on full-size and ``toy'' versions of block ciphers. In this section, we describe some effective attacks that exploit specific properties of the S-box during the attack. Such attacks have already been studied by Kaminsky et al. \cite{Kaminsky10-2} in the context of the AES, who cover this topic in more breadth with additional information about side channel attacks, for example. We refer the reader to their survey for more information on recent AES-specific cryptanalysis results. In this work, we use the following attacks as motivation for measuring the strength of S-boxes, which can be perceived as their relative resistance to these attacks. As we will show in Chapter \ref{chp:sboxConstruction}, these metrics are then used when designing cryptographically strong S-boxes. We stress that this list of attacks is by no means exhaustive; it simply contains some of the most important published attacks. For the purposes of this work, we chose to focus on this particular subset as they are, in a sense, the most known.

\subsection{Linear Cryptanalysis}
Since S-boxes are typically the only source of nonlinearity and, consequently, confusion, in an SPN design, it is critically important to understand the degree to which they can be approximated as linear equations \cite{stinson-crypto}. For the purposes of this section, we consider only bijective S-boxes $S : \{0,1\}^n \to \{0,1\}^n$. In the context of linear cryptanalysis, we say that for each input element of the S-box, which may be viewed as an $n$-dimensional vector of coordinates $(x_0, x_1,\dots,x_{n-1})$, there is a corresponding set of $n$ independent random variables $\mathbf{X}_i$, $0 \leq i < n$. Similarly, for each output element of the S-box $\bar{y} = (y_0, y_1,\dots,y_{n-1})$, there are $n$ corresponding random variables $\mathbf{Y}_i$, $0 \leq i < n$. However, these variables are not specifically independent from each other or from the $\mathbf{X_i}$ variables, since the probability of the output depends on the input.

The underlying goal of linear cryptanalysis is to find and exploit some linear combination $\mathbf{X_{i,1}} \oplus \mathbf{X_{i,1}} \oplus \dotsb \oplus \mathbf{X_{i,s}} = \mathbf{Y_{j,1}} \oplus \mathbf{Y_{j,2}} \oplus \dotsb \oplus \mathbf{Y_{j,t}}$ that is satisfied with a high (or low) probability. For an ideal SPN, such relationships will be satisfied exactly half of the time for any selection of random variables $\mathbf{X_{i,1}},\dots,\mathbf{X_{i,s}},\mathbf{Y_{j,1}},\dots,\mathbf{Y_{j,t}}$. Should there exist some selection of random variables for a linear combination such that the probability of satisfying the relationship is not $1/2$, then this deviation, or \emph{bias}, from the relationship can be exploited in a linear cryptanalysis attack. With this goal as motivation, we begin our discussion of linear cryptanalysis with some more formal definitions.
\begin{definition}
Let $p_i$ be the probability that $\mathbf{Pr}[\mathbf{X_i} = 0]$. We define the \emph{bias} of $x_i$, denoted $\epsilon_i$, as the quantity:
\begin{align*}
\epsilon_i = p_i - \frac{1}{2}.
\end{align*}
% Note that $p_i$ is equivalent to the probability that this linear combination is satisfied.
\end{definition}
Now, we consider all possible $2^n$ random variable combinations for $n$ variables. For an S-box with a 4-bit domain and range, we have a total of $8$ random variables to examine, which, together, correspond to the $4$ input and $4$ output bits. Therefore, there are $2^8$ possible combinations that can be used to calculate biases. Let the $n$-dimensional vectors $\mathbf{a} = ( a_1,a_2,\dots,a_n )$ and $\mathbf{b} = ( b_1,b_2,\dots,b_n )$, where $a_i, b_i \in \{0,1\}$ for all $1 \leq i \leq n$ correspond to these input and output elements for the S-box. We can represent all $2^8$ linear combinations as follows
\begin{align*}
\left(\bigoplus_{i=i}^{n}a_i\mathbf{X_i}\right) \oplus \left(\bigoplus_{i=i}^{n}b_i\mathbf{Y_i}\right) = 0,
\end{align*}
or, equivalently,
\begin{align*}
\left(\bigoplus_{i=i}^{n}a_i\mathbf{X_i}\right) = \left(\bigoplus_{i=i}^{n}b_i\mathbf{Y_i}\right).
\end{align*}
Using the Piling-Up Lemma presented by Matsui \cite{Matsui91-1}, we have 
\begin{align*}
Pr[\mathbf{X_i} \oplus \dotsb \oplus \mathbf{X_n} = 0] = \frac{1}{2} + 2^{n-1} \prod_{i=1}^{n}\epsilon_i,
\end{align*}
which means that 
\begin{align*}
\epsilon_{1,2,\dots,n} = 2^{n-1}\prod_{i=1}^{n}\epsilon_i,
\end{align*}
where $\epsilon_{1,2,\dots,n}$ is the bias of $\mathbf{X_i} \oplus \dotsb \oplus \mathbf{X_n}$.
This leads to the fundamentally simple (albeit very clever) counting-based attack that is linear cryptanalysis.
More specifically, if we use a counting-based method for determining the bias for all possible linear
combinations of input and output variables $\mathbf{X_i}$ and $\mathbf{Y_j}$, we can 
identify input and output variables that are suitable candidates for conducting a linear
cryptanalysis attack. Treating the $\mathbf{a}$ and $\mathbf{b}$ vectors as binary numbers $a$ and $b$,
we may tabulate $N_L(a,b)$, the number of tuples
$(x_1,x_2,...,x_n,y_1,y_2,...,y_n)$ such that $(y_1,y_2,...,y_n) = S_S(x_1,x_2,...,x_n)$ and 
\begin{align*}
\left(\bigoplus_{i=i}^{n}a_ix_i\right) \oplus \left(\bigoplus_{i=i}^{n}b_iy_i\right) = 0.
\end{align*}
With the $N_L(a,b)$ table, we then compute the bias $\epsilon_{1,2,\dots,n}$ by $(N_L(a,b) - 2^{n-1}) / 2^n$, 
and from this, the probability $p_i$ that the linear combination $\mathbf{X_i} \oplus \dotsb \oplus \mathbf{X_n} = 0$
was satisfied, i.e. $p_i = 1/2 - \epsilon_{1,2,\dots,n}$. 

Using the example linear cryptanalysis attack presented in \cite{stinson-crypto}, we give a generalized procedure for realizing this type of attack in Algorithm \ref{alg:linearAttack}. We borrow the same notation from Stinson in which plaintext elements $x$ and ciphertext elements $y$ are bit strings of length $lm$, which can be viewed as the concatenation of $l$ separate $m$-bit strings. With this notation, we refer to the $i$th block of length $m$ in a plaintext (ciphertext) element $x$ ($y$) as $x_i$ ($y_i$), $0 \leq i < l$. In our procedure we use $\mathcal{P}$ to denote the set of all plaintext and ciphertext pairs collected prior to evaluation, and $\mathcal{B}_x$ and $\mathcal{B}_y$ are sets of block (random variable) indexes that are used in the predetermined linear combination of S-box input and output elements, respectively. That is, the indexes in $\mathcal{B}_x$ and $\mathcal{B}_y$ correspond to a selection of random variables corresponding to a linear combination of input and output bits that is satisfied with relatively high or low probability, as determined from the $N_L(a,b)$ table. Finally, we denote the inverse S-box as $S_{-1}$ and the set of all possible candidate keys as $\mathcal{K}$, where for each $K \in \mathcal{K}$ we have that $|K| = |x| = |y| = lm$.

\begin{algorithm}[t!] %[htb]
% Bx and By are the bits in x and y that are in the linear relationship, resp.
\caption{General Linear Cryptanalysis Attack} \label{alg:linearAttack}
\begin{algorithmic}[1]
\Require $\mathcal{P}, \mathcal{K}, \mathcal{B}_x, \mathcal{B}_y, \mathcal{S}^{-1}$, $l$, $m$
\State $Count[K] \gets [0...2^{lm}]$
\For {\textbf{each} $K \in \mathcal{K}$}
	\State $Count[K] \gets 0$
\EndFor
\For{\textbf{each} $(x, y) \in \mathcal{P}$}
	\For {$K \in \mathcal{K}$}
		\State $V \gets []$
		\For {$i = 1 \text{ to } |\mathcal{B}_y|$}
			\State $v \gets \mathcal{S}^{-1}(K \oplus y_{\mathcal{B}_{y}[i]})$
			\State $V = Append(V, v)$
		\EndFor
		\State $z \gets \bigoplus_{i = 1}^{|\mathcal{B}_x|} x_{\mathcal{B}_{x}[i]} \bigoplus_{i=1}^{|\mathcal{B}_y|} V[i]$ 
		\If{$z = 0$}
			\State $Count[K] \gets Count[K] + 1$
		\EndIf
	\EndFor
\EndFor
\State $max \gets -1$
\State $K^* = \{0\}^{lm}$
\For {\textbf{each} $K \in \mathcal{K}$}
	\If{$Count[K] > max$}
		\State $max \gets Count[K]$
		\State $K^* \gets (K)$
	\EndIf
\EndFor
\State \Return $K^*$
\end{algorithmic}
\end{algorithm}

As previously stated, the goal of linear cryptanalysis is to find a set of linear approximations of active S-boxes that help approximate an entire SPN algorithm throughout all of the rounds (with the exception of the last). Matsui \cite{Matsui91-1} showed that the complexity of this known-plaintext attack (i.e. the number of plaintexts required for a successful attack) is proportional to $\epsilon^{-2}$, where $\epsilon$ is the bias from $1/2$ that a linear expression exhibited for the entire SPN-like block cipher. It is generally known that larger S-box biases correspond to larger biases for every round of the block cipher, which is expected since they are typically the only nonlinear components of the algorithm. Therefore, studying the resiliency of S-boxes to this type of attack is critically important in the design of cryptographically strong S-boxes. In Section \ref{sec:strength} we will discuss the properties of S-boxes in SPN designs that improve the algorithm's resiliency to this type of attack.

\subsection{Differential Cryptanalysis}
Differential cryptanalysis, first introduced as an attack on the Data Encryption Standard \cite{Biham91-1}, is a chosen plaintext attack that attempts to find and exploit certain occurrences of input and output differences in the last round of a cipher that occur with a high probability \cite{stinson-crypto}. Put another way, for an ideally randomizing block cipher with a sufficiently strong S-box, the probability that a certain output difference $\Delta Y = Y_i \oplus  Y_j$ will occur given an input difference $\Delta X = X_i \oplus X_j$ is exactly $1/2^n$ (where $n$ denotes the number of bits in the input $X$). For future reference, the pair $(\Delta X, \Delta Y)$ is referred to as a differential. Thus, by finding output differences that occur with high probability for each round of a target cipher, one can establish a relationship between the plaintext and input to the last round of the cipher. Such a relationship is loosely referred to as a \emph{differential trail} \cite{Daemen02-1}. Then, by sampling a large number of plaintext and ciphertext pairs, the attacker can guess the key by counting the number of times a given differential trail holds for a candidate key. In ciphers where the round key schedule is invertible, this enables the attacker to uncover the secret key. Clearly, the nonlinearity of the S-box plays a pivotal role in the establishment of the differential characteristic for the entire cipher. Also note that in SPN designs, the key does not influence the coordinates of a differential. For example, given the differential $(\Delta X, \Delta Y)$, we have the following:
\begin{align*}
\Delta Y & = Y_i \oplus Y_j \\
& =  (X_i \oplus K) \oplus (X_j \oplus K) \\
& =  X_i \oplus X_j \oplus K \oplus K \\
& =  X_i \oplus X_j \\
& = \Delta X
\end{align*} 
To carry out this attack, the attacker must collect a large sample of plaintext and corresponding ciphertext pairs $(X,Y)$. Given that the S-box is the key to preventing this attack, the attacker then computes a difference distribution table $N_D$, where
\begin{align*}
N_D(\Delta X, \Delta Y) = |\{(X_i, X_j) \in \Delta X : S(X_i) \oplus S(X_j) = \Delta Y\}|.
\end{align*}
One may observe that in an ideal S-box, $N_D(\Delta X, \Delta Y) = 1$ for all $\Delta X$ and $\Delta Y$. However, this is not possible with bijective S-boxes because $N_D(\Delta X, \Delta Y = \Delta X) = 2^n$. 

We denote the \emph{propagation ratio} $R_D(\Delta X, \Delta Y) = N_D(\Delta X, \Delta Y) / 2^n$, which may be interpreted as the probability that an output difference $\Delta Y$ occurs given a certain input difference $\Delta X$. Now assume that for a certain round $r$ of a SPN cipher we find a differential $(\Delta X, \Delta Y)$ with a high propagation ratio, and further assume that $\Delta Y$ is equal to $\Delta X'$ in a differential $(\Delta X', \Delta Y')$ with a high propagation ratio in round $r + 1$. We may then combine these differentials together, forming a subset of the previously mentioned differential trail, and compute the resulting propagation ratio as $R_D(\Delta X, \Delta Y) \cdot R_D(\Delta X', \Delta Y')$. By continuing this process, we may obtain the propagation ratio for a differential trail of the all $N_r - 1$ rounds of the block cipher (up to the last round). The attacker can then use this differential trail to determine which candidate keys satisfy the input and output difference with the highest probability. A generic description of this process is given in Algorithm \ref{alg:dcattack}. Note that we now require a new parameter $\mathcal{B}_t$ that contains the indexes that of the output elements $y$ and $y'$ such that $y_i = y'_i$ for all $i \in \mathcal{B}_t$. Counting keys that do not satisfy this constraint would introduce random noise into the attack, and should thus be avoided.

\begin{algorithm}[t!] %[htb]
\caption{General Differential Cryptanalysis Attack} \label{alg:dcattack}
\begin{algorithmic}[1]
\Require $\mathcal{P}, \mathcal{K}, \mathcal{B}_x, \mathcal{B}_y, \mathcal{B}_t, \mathcal{S}^{-1}, l, m$
\For {\textbf{each} $K \in \mathcal{K}$}
	\State $Count[K] \gets 0$
\EndFor
\For{\textbf{each} $(x, y, x', y') \in \mathcal{P}$}
	\State $R \gets True$
	\For {$i = 1 \text{ to } |\mathcal{B}_t|$}
		\If {$y_{\mathcal{B}_t[i]} \not= y'_{\mathcal{B}_t[i]}$}
			\State $R \gets False$
		\EndIf
	\EndFor
	\If {$R = True$}
		\For {\textbf{each} $K \in \mathcal{K}$}
			\State $D \gets []$
			\For {$i = 1 \text{ to } |\mathcal{B}_y|$}
				\State $v \gets \mathcal{S}^{-1}(K \oplus y_{\mathcal{B}_{y}[i]})$
				\State $v' \gets \mathcal{S}^{-1}(K \oplus y'_{\mathcal{B}_{y}[i]})$
				\State $D \gets Append(D, v \oplus v')$
			\EndFor
			\State $match \gets 1$
			\For{$i = 1 \text { to } |\mathcal{B}_y|$}
				\State $match \gets match \land (D[i] = \Delta Y^*)$ 
			\EndFor
			\If{$match = 1$}
				\State $Count[K] \gets Count[K] + 1$
			\EndIf
		\EndFor
	\EndIf
\EndFor
\State $max \gets -1$
\State $K^* \gets \{0\}^{lm}$
\For {\textbf{each} $K \in \mathcal{K}$}
	\If{$Count[K] > max$}
		\State $max \gets Count[K]$
		\State $K^* \gets (K)$
	\EndIf
\EndFor
\State \Return $K^*$
\end{algorithmic}
\end{algorithm}

%\section{Differential-Linear Cryptanalysis}
%TODO

\subsection{XL and XLS Algebraic Attacks}
Thanks to Courtois et al. \cite{Courtois00-1}, we know that it is possible to represent block ciphers as overdefined systems of multivariate quadratic (MQ) equations $A$, where each equation $l_k \in A$ is of the form
\begin{align*}
l_k & = \sum_{i,j}a_{i,j,k}x_ix_k + \sum_{i}b_{i,k}x_i + c_k = 0,
\end{align*}
and $a_{i,j,k},b_{i,k},c_k \in GF(2)$. Solving such a system is known to be an NP-hard \cite{Fraenkel79-1} problem. However, solving a system of linear equations can be done in polynomial time using techniques such as Gaussian elimination. In 2000, Shamir, Courtois, Klimov, and Patarin \cite{Courtois00-1} introduced a technique known as eXtended Linearization (XL), which transforms a MQ problem of $m$ equations and $n$ variables into a (much) larger system of solvable linear equations. This technique is an extension of the ``linearization'' technique first introduced in \cite{Kipnis99-1}.

The procedure works by first generating a set of $D - 2$ ($D \geq n/\sqrt{m}$) new variables with all powers less than or equal to $D - 2$. Referring to the example given in \cite{Kleiman05-1}, if we have a set of variables $\{x, y, z\}$ with $D = 4$, then the resulting set of variables is $\{x,y,z,x^2,y^2,z^2,xy,xz,yz\}$. Each variable in this set is then multiplied by the original $m$ equations. Then, for each resulting equation, the individual monomial terms are all replaced by a distinct new variables. For example, for an equation $l_i = x^3yz + xy = 0$, we would generate a corresponding equation $l_i = a + b = 0$, where $a = x^3yz$ and $b = xy$, respectively. From here, we have a set of linear equations that can be solved in polynomial time. Of course, the complexity of this process depends greatly on the selection of $D$ and the original $n$ and $m$ parameters in the MQ system. Since the algebraic expression of SPN ciphers has a direct impact on the form of the MQ system, it is clear that improving the complexity of this expression helps thwart the success of this attack. A formal description of the XL algorithm is shown in Algorithm \ref{alg:xl}.

\begin{algorithm}[t!] %[htb]
\caption{Extended Linearization} \label{alg:xl}
\begin{algorithmic}[1]
\Require $\mathbb{F}, m, n, A = \{l_1,\dots,l_r\}$
\State Generate all polynomial equations $l_i' = \left(\prod_{i,j}^k x_{i,j}\right) l_i$, where $k \leq D - 2$. 
\State $varMap = \{\}$
\For{$l_i' = l_1' \text{ to } l_k'$}
	\For{$j = 1 \text{ to } |l_i'|$} \Comment{For each term in $l_i'$}
	% \For{ \textbf{each } $p_j \in l_i'$}
		\If {$p_j \not\in varMap$}
			\State $v = |varMap| + 1$ \Comment{Generate a new variable not in the map}
			\State $varMap[v] = l_i'[j]$
		\EndIf
		\State $l_i'[j] = varMap[l_i'[j]]$ \Comment{Perform substitution}
	\EndFor
\EndFor
\State Perform Gaussian elimination for the modified system of equations $\{l_1',\dots,l_r'\}$.
\While{There exists at least one univariate equation $l_i'$ in the powers of $x_i$}
	\State Solve $l_i'$ over the field $\mathbf{F}$
	\State Simplify the set of equations $\{l_1',\dots,l_r'\}$
\EndWhile
\State Perform backwards substitution to find the original variables
\end{algorithmic}
\end{algorithm}

To date, the XL attack has failed to break popular ciphers such as Rijndael and Serpent. However, Courtois and Pieprzyk \cite{Courtois02-1} discovered a way to modify the XL attack to account for the sparse set of equations that are introduced in the process. This variation, coined as the eXtended Sparse Linearization (XSL) attack, is different in that the process by which new variables are defined and then subsequently multiplied by the $m$ equations in the MQ system is modified to only work with a select set of monomials. This reduces the overall size of the resulting equations, and thus simplifies the solving process. Courtois and Pieprzyk noticed that this select set of monomials in the multiplication step corresponds to those that already appear in other equations (i.e. they carefully select monomials that do not introduce new variables into the resulting $l_i'$ equations). The most important implication of this attack is that it \emph{does not} grow with the number of rounds in the cipher, as is the case with the traditional linear and differential cryptanalysis attacks. Rather, the complexity is based solely on the structure of the cipher, and more specifically, on the complexity and degree of the S-box algebraic expression.

%\cite{Cryptanalysis of Block Ciphers with Overdefined Systems of Equations}

\subsection{Interpolation Attacks} \label{sec:sublayer_interpolation}
Interpolation attacks consider the problem of representing a cipher as a high-order polynomial with key-dependent coefficients \cite{Jakobsen97-1}. The intuition behind this representation is that by then solving for these coefficients it is possible to deduce the secret key. Given a field $GF(2^k)$, such a polynomial $p(x)$ is obtained using Lagrangian interpolation. In particular, with $n = 2^k$ distinct input elements $x_1,\dots,x_k$ and output elements $y_1,\dots,y_k$, we may determine a polynomial $p(x)$ by
\begin{align*}
p(x) = \sum_{i=1}^{n} y_i \prod_{1 \leq j \leq n, j \not= i} \frac{x - x_j}{x_i - x_j}.
\end{align*}
Jakobsen et al. \cite{Jakobsen97-1} prove that for a given block cipher of block size $l$, one may express the ciphertext as a polynomial of the plaintext, where $m$ is the number of coefficients in the polynomial. If $l < 2^m$, then it is possible to conduct an interpolation attack to find a polynomial to solve for computing the output ciphertext of the cipher using input with a fixed number of bits. Furthermore, this process can be carried out in $\mathcal{O}(l)$ time using $l$ plaintexts encrypted with the same secret key. The authors then extended this technique to solve for the actual secret key. Their attack is generalized in the following theorem.
\begin{thm}
\cite{Jakobsen97-1} Given an iterated block cipher of block size $l$, it is possible to express the ouput from round $N_r - 1$ as a polynomial of the plaintext with $m$ coefficients. With $b$ key bits in the last round, there exists an interpolation attack of time complexity $\mathcal{O}(2^{b - 1}(l + 1))$ that requires $l + 1$ known plaintexts which can successfully deduce the secret key. 
\end{thm}
In the case of Rijndael, the polynomial used to represent the S-box is:
\begin{align*}
p(x) = 05x^{254} + 09x^{253} + F9x^{251} + 25x^{247} + F4x^{239} + 01x^{223} + B5x^{191} + 8Fx^{127} + 63,
\end{align*}
where all coefficients are hexadecimal numbers, and thus elements of the field $GF(2^8)$. This algebraic expression was obtained using Lagrangian interpolation \cite{Daemen02-1} (this technique is discussed in more detail in Chapter 3). Since the degree of $p(x)$ is $9$, and so this class of interpolation attack is not feasible. However, the algebraic simplicity of Rijndael's S-box, which is directly correlated to the complexity of the polynomial representation for the entire cipher, leaves many cryptographers skeptical about its security and future susceptibility to related attacks.

% \subsection{Correlation Attacks}
% Correlation attacks were first introduced on nonlinear combination generators, which 
% are composed of $m$ independent single-bit output LFSRs whose output forms $m$ bits
% of the keystream in a stream cipher. The attack worked by exploiting existence of 
% a statistical dependence between the output of the generator (combining function) 
% and the output of one of the component LFSRs \cite{Canteaut05-1}. That is,
% \begin{align*}
% \mathbf{Pr}[f(x_1,\dots,x_m) \not= x_i] \not= \frac{1}{2},
% \end{align*}
% where $x_i$ is the output of the $i$th component LFSR and $f$ is the combination function.
% This attack requires knowledge of the keystream bits *****
