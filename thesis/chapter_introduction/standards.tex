\section{Symmetric-Key Cryptosystem Standards}
The Data Encryption Standard (DES), which was officially standardized in the United States
by the National Bearue of Standard (NBS) in 1977, was monumental in the advancement of
the understanding modern cryptgraphy and cryptanalysis attacks dealing with block ciphers (TODO: CITE FIPS).
The algorithm itself is built as a Feistel (named after Horst Feistel of IBM) SPN. 

TODO: continue with discussion of DES (with emphasis on properties of the S-box), and then AES (with emphasis on
the S-box)

\subsection{Data Encryption Standard}
- DES (SPN with description)

\subsection{Advanced Encryption Standard}
Rijndael, the winner of the Advanced Encryption Standard contest in 2001 \cite{daemen01-AES}, is today's 
most widely used symmetric-key block cipher. Rijndael is a SPN product cipher that achieves confusion 
and diffusion through four fundamental oprations: AddRoundKey, ShiftRows, SubBytes $(S)$, and MixColumns. 

TODO: describe the algorithm here...

A single round in Rijndael (with the exception of the last round) consists of the application
of these four operations in the following manner:
\begin{align*}
\left( \begin{array}{c}
s_{1,c} \\
s_{2,c} \\
s_{3,c} \\
s_{4,c} \end{array} \right) = S
\end{align*}

It is known that the Rijndael S-box, also referred to as the SubBytes operation, is the only source of non-linear behavior in the algorithm. Formally, it is defined as a permutation of the bytes (or bundles) of the state, and is commonly denoted by $S_{RD}$. 

When constructing this specific S-box, the authors chose to strive for the following characteristics:
\begin{itemize}
	\item \textbf{Non-linearity}
	\begin{itemize}
		\item \textbf{Correlation.} The maximum input-output correlation amplitude must be as small as possible.
		\item \textbf{Difference propagation probability.} The maximum difference propagation probability must be as small as possible.
	\end{itemize}
	\item \textbf{Algebraic complexity.} The algebraic expression of $S_{RD}$ in $GF(2^8)$ must be complex.
\end{itemize}

Justification of the choice of S-box is based on published attacks at the time of the NIST competition and on the selection critera specified by Nyberg in \cite{Nyberg1994-DUM}. This paper will be the next point of reference in the following week. When finished, I will revisit the Rijndael book.

Any 8×8-bit S-box can be considered as a composition of 8 Boolean functions
sharing the same 8 input bits. J. Fuller and W. Millan observed that the S-box
of Rijndael can be described using one Boolean function only [8]. The 8 Boolean
functions can be described as
\begin{align*}
f_i(x_1,\dots,x_8) = f(g_i(x_1,\dots,x_8)) + c_i,
\end{align*}
where $g_i$ and $c_i$ are affine transformations and constants for the $i = 1,\dots,8$.

``On linear redundancy in S-boxes''