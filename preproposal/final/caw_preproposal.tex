\documentclass[10pt]{article}
\usepackage{fullpage}
\usepackage{palatino}
\usepackage{bbm}
\usepackage{url}
\usepackage{verbatim}
%\usepackage{cawsty}

\usepackage{thumbpdf, amssymb, amsmath, amsthm, microtype,
	    graphicx, verbatim, listings, color, fancybox}
\usepackage[pdftex]{hyperref}
%\usepackage[margin=1in]{geometry}
\usepackage{fullpage}
\usepackage{pseudocode}
\usepackage{fancybox, fancyvrb}

\newcommand{\field}[1]{\mathbb{#1}} %requires amsfonts

\begin{document}
\thispagestyle{empty} 
\begin{center}
{\em MS Thesis Preproposal}\\
{\large \bf Large Substitution Boxes with Efficient Implementations}\\
{\bf Christopher A. Wood}\\
{\bf \it Committee Chair: Professor Stanis{\l}aw Radziszowski}\\
Department of Computer Science, Rochester Institute of Technology\\
%\today{}\\
\end{center}

%TODO: finish and send to SPR/ML/Mike for review on 3/7/13
\vspace{-1em}
\section{Problem Statement}
S(ubstitution)-boxes are the most common nonlinear building block of symmetric-key
block ciphers. An S-box can be defined as a function $F : GF(2^n) \to GF(2^m)$. To measure
the cryptographic strength of these functions it is common to represent them as vectorial Boolean functions,
where $F(x) : (f_1(x), \dots, f_m(x))$ and $f_i : GF(2^n) \to GF(2)$. Such a representation
enables one to measure its nonlinearity (i.e. measure of distance from affine functions), algebraic
immunity (i.e. difficulty measure of annihilator-based algebraic attacks), resiliency (i.e. balancedness 
and correlation immunity between input and output of the function), and differential uniformity (i.e.
difficulty measure of differential cryptanalysis).

Naturally, S-box security comes at the price of computational efficiency.
In resource constrained systems, such as VLSI circuits and embedded platforms, traditional LUT-based
implementations of S-boxes are not feasible. In such systems, 
combinational logic circuits or software functions are commonly used to implement the S-box.
For this reason, cryptographers have worked for many years to find a balance between the strength
and efficiency of S-boxes. In addition, with the selection of the Advanced Encryption Standard,
all of this research has been focused on 8-bit S-boxes (i.e. $F : GF(2^8) \to GF(2^8)$).
%While this decision was seemingly made so as to match the domain and range of $F$ to the size of the
%state elements in Rijndael (i.e. a byte), it may be worthwhile to extend this definition to higher-order
%S-boxes. 
%With such a modification to the design of S-boxes, resource constraints are exacerbated and 

To our knowledge, there has not been any work that studies the strength and
implementation efficiency (in terms of hardware area or total gates required) of higher order
S-boxes. Therefore, in this work, we break away from tradition and focus our attention on $16$-bit
S-boxes defined as $F : GF(2^{16}) \to GF(2^{16})$. The problem is to study these S-boxes
in the context of Boolean functions to determine their strength, and for each ideal candidate
S-box, attempt to find optimal hardware and software implementations either through the use of Boolean
function minimization or composite field arithmetic.
\vspace{-1em}
\section{Proposed Approach}
The first part of our research will focus on the cryptographic strength of 
various S-box definitions using analysis methods for Boolean functions. Clearly, exhaustively searching
all $GF(2^{16^{16}})$ S-box definitions is infeasible, so our analysis will be constrained
to S-boxes defined by operations in Galois Fields (i.e. $F(x) = F(x)^{-1}$, $x \in GF(2^{16})$)
and built using the Maiorana-McFarland Boolean function construction technique. Time
permitting, we may also explore Boolean function construction techniques targeted towards specific
cryptographic properties, such as resiliency and algebraic immunity. 
The metrics collected for this part of the research include the nonlinearity,
algebraic immunity, resiliency, and differential uniformity of each S-box candidate. 

%Some of these metrics can be gathered using third-party software tools like 
%SAGE and Mathematica, while others will require custom software to measure.
 
The second thread of our research will be to implement candidate S-boxes in FPGA hardware. Software versions
of these functions will also be implemented and optimized to measure the throughput performance
on low-end platforms (i.e. 32-bit PowerPC hardcore processors and 8-bit Atmel AVR microcontrollers). 
The metrics collected for this part of the research include hardware area (i.e.
FPGA slices and total synthesized logic gates) and throughput (cycles per byte), as well
as the software memory footprint for S-box functions and their throughput (cycles per byte).
%Power consumption will not be measured.
\vspace{-1em}
\section{Evaluation}
To evaluate our work we will first formalize our experimental methodology and testbed for 8-bit S-boxes and 
compare the collected metrics against those published in the literature. Once verified for correctness,
we will then scale up the S-boxes to 16-bits. Since this is a new line of research, we have
no benchmark against which to compare our results. The primary outcome of this thesis will be 
the establishment of such a benchmark.

\end{document}
