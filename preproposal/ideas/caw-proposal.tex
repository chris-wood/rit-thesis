\documentclass[11pt]{article}

\usepackage{thumbpdf, amssymb, amsmath, amsthm, microtype,
	    graphicx, verbatim, listings, color, fancybox}
\usepackage[pdftex]{hyperref}
%\usepackage[margin=1in]{geometry}
\usepackage{cawsty}
\usepackage{fullpage}
\usepackage{pseudocode}
\usepackage{verbatim}

\newcommand{\tlg}{\text{ lg}}
\newcommand{\tln}{\text{ ln}}

%\setlength{\parindent}{0pt}

\linespread{1.2}

\begin{document}
\cawtitle{Optimizing Diffusion and Confusion}{in Cryptographic Primitives}{M.S. Thesis Proposal}

%\begin{abstract}
%TODO
%\end{abstract}

\section{Objectives}
The following list enumerates some long-term goals from this research effort.

% the objective list
\begin{itemize}
	\item (THEORY) Study the mathematical theory behind diffusion and confusion, stemming from Shannon's communication entropy and the use of nonlinear functions in cryptographic primitives.
	\begin{itemize}
		\item Question: Can discrete-time dynamical system equations be utilized effectively as diffusion layers? \cite{ChaosCrypto}
		\item Question: To what extent does the study of chaos theory lend itself to constructing effective diffusion layers in cryptographic primitives? \cite{ChaosCrypto}
		\item Question: Are there any other unpopular or unused discrete mathematical objects that can be used to build (fast and secure) diffusion layers with high nonlinearity properties? \cite{cryptoeprint:2010:579} \cite{cryptoeprint:2010:362} \cite{Nyberg:1991:PNS:1754868.1754910}
	\end{itemize}
	\item (THEORY) Analyze and evaluate existing cryptographic algorithm design mechanics that promote high diffusion and confusion through nonlinearity.
	\begin{itemize}
		\item Substitution-Permutation Network designs (SPN).
		\begin{itemize}
			\item Skein
		\end{itemize}
		\item Add-Rotate-XOR (ARX) designs.
		\begin{itemize}
			\item Rijndael
		\end{itemize}
		\item Question: Can existing sources of nonlinearity (S-boxes, ARX functions, etc) be manipulated to provide higher measures of diffusion without affecting their susceptibility to linear and differential cryptanalysis?
		\item Question: What diffusion layer designs provide the best security and performance tradeoffs? Can this be mathematically proved or solved by reduction to an integer optimization problem?
	\end{itemize}
	\item (THEORY/SOFTWARE) Investigate the application of combinatorial and integer optimization techniques to block cipher designs and internal operations.
	\begin{itemize}
		\item Question: Can diffusion layers be represented as multivariate polynomial equations, and if so,  can we apply common optimization techniques to find the most effective constructions?
		\item Question: How can such optimization techniques be extended to the cryptanalysis of block ciphers?
	\end{itemize}
	\item (THEORY/SOFTWARE) Perform a statistical and randomness analysis on existing diffusion layers in cryptographic primitives.
	\begin{itemize}
		\item Question: To what extent can the statistical properties of diffusion layers be extrapolated to full-round cryptographic primitives?
	\end{itemize}
	\item (SOFTWARE/HARDWARE) Investigate the implementation aspects of chosen diffusion layers
	\begin{itemize}
		\item Explore optimization techniques for different software and hardware implementation mediums (but limit to reconfigurable hardware - FPGAs) \cite{CryptoEngineering} \cite{henriquez2006cryptographic}
	\end{itemize}
\end{itemize}

\bibliography{caw-proposal}{}
\bibliographystyle{plain}

\end{document}
